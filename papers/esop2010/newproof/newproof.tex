\documentclass[a4paper]{article}

\usepackage[utf8x]{inputenc}
\usepackage[T1]{fontenc} 
\DeclareUnicodeCharacter{1013}{\ensuremath{\epsilon}}
\newcommand{\textalpha}{\ensuremath{\alpha}}
\providecommand{\textbeta}{\ensuremath{\beta}}
\newcommand{\textchi}{\ensuremath{\chi}}
\newcommand{\textdelta}{\ensuremath{\delta}}
\newcommand{\textepsilon}{\ensuremath{\varepsilon}}
\newcommand{\texteta}{\ensuremath{\eta}}
\newcommand{\textiota}{\ensuremath{\iota}}
\newcommand{\textkappa}{\ensuremath{\kappa}}
\newcommand{\textlambda}{\ensuremath{\lambda}}
\providecommand{\textmu}{\ensuremath{\mu}}
\newcommand{\textomega}{\ensuremath{\omega}}
\newcommand{\textpi}{\ensuremath{\pi}}
\newcommand{\textphi}{\ensuremath{\varphi}}
\newcommand{\textrho}{\ensuremath{\rho}}
\newcommand{\textomicron}{\ensuremath{o}}
\newcommand{\textsigma}{\ensuremath{\sigma}}
\newcommand{\texttau}{\ensuremath{\tau}}
\newcommand{\texttheta}{\ensuremath{\theta}}
\newcommand{\textxi}{\ensuremath{\xi}}
\newcommand{\textzeta}{\ensuremath{\zeta}}

\newcommand{\textLambda}{\ensuremath{\Lambda}}
\newcommand{\textDelta}{\ensuremath{\Delta}}
\newcommand{\textGamma}{\ensuremath{\Gamma}}
\newcommand{\textPi}{\ensuremath{\Pi}}
\newcommand{\textPhi}{\ensuremath{\Phi}}
\newcommand{\textSigma}{\ensuremath{\Sigma}}

\usepackage{stmaryrd}
\usepackage{amsmath,amssymb}
\usepackage{amsthm}
\usepackage{mathpartir}
\usepackage{url}

\newtheorem{thm}{Theorem}[section]
\newtheorem{lem}[thm]{Lemma}
\theoremstyle{definition}
\newtheorem{definition}[thm]{Definition}


\newcommand{\wpre}{{\bf wp}}
\newcommand{\sep}{ ~|~ }
\newcommand{\letml}{{\bf let}}
\newcommand{\inml}{{\bf in}}
\newcommand{\ifml}{{\bf if}}
\newcommand{\thenml}{{\bf then}}
\newcommand{\elseml}{{\bf else}}
\newcommand{\refml}{{\bf ref}}
\newcommand{\preml}{{\bf pre}}
\newcommand{\postml}{{\bf post}}
\newcommand{\propml}{{\bf prop}}
\newcommand{\boolml}{{\bf bool}}
\newcommand{\recml}{{\bf rec}}
\newcommand{\efft}[1]{\langle #1 \rangle}
\newcommand{\lift}[1]{\langle #1 \rangle}
\newcommand{\alist}[1]{\overline{#1} }
\newcommand{\Trueml}{{\bf True}}
\newcommand{\Falseml}{{\bf False}}
\newcommand{\trueml}{{\bf true}}
\newcommand{\falseml}{{\bf false}}
\newcommand{\unitml}{{\bf unit}}
\newcommand{\intml}{{\bf int}}
\newcommand{\coq}{Coq}
\newcommand{\who}{Who}
\newcommand{\ocaml}{Ocaml}
\newcommand{\correct}[1]{c(#1)}

\newcommand{\ceil}[1]{\lceil #1 \rceil}

\title{A proof}
\author{Johannes Kanig \and Jean-Christophe Fill\^atre}
\begin{document}
\maketitle

\section{Definitions}

\begin{figure}[tpb]
\begin{eqnarray*}
  x & & \text{Program Variables}\\
  α & & \text{Type Variables}\\
  e & & \text{Effect Variables}\\
  ρ & & \text{Region Variables}\\
  r & & \text{runtime regions}\\
  χ &::=& α \sep e \sep ρ \\
  t,p,q &::=& r \sep x~[\alist{κ}] \sep λx:τ.t \sep \recml~f~(x:τ).\{p\}t\{q\} \sep t~t\sep \\ 
  & & \letml~ x~[\alist{χ}] = t~\inml~t \sep \ifml~t~\thenml~t~\elseml~t \sep
  \letregionml~ρ~\inml~t\\
  τ &::=& α \sep τ -> τ \sep τ ->^ε τ \sep τ \times τ \sep τ~\refml_ρ \sep
  \efft{ε} \\
  ε &::=& (ρ \sep r \sep e)*\\
  κ &::=& τ \sep ε \sep ρ
\end{eqnarray*}
  \caption{Syntax}
  \label{fig:syntax}
\end{figure}

\begin{figure}[tpb]
\begin{equation*}
   \begin{array}[t]{rrcl}
    β & s, (λ x:τ.e)\ v & --\ & s, e [x|->v] \\
    \recml & s, (\recml\ f\ (x : τ). v_1)\ v_2 & --\ & s, v_1[f |->
    (\recml~f~(x:τ).v_1), x |-> v_2]\\
    let & s, \letml~ x~\alist{χ} = t~\inml~t & --\ & s, e[x|->Λ\alist{χ}.v] \\
    ref & s, \refml~v & --\ & s[r|->v], r \qquad r~{\textup fresh}\\
    := & s, r := v & --\ & s[r|->v], () \qquad r\in dom(s) \\
    ! & s, !r & --\ & s, s(r) \qquad r\in dom(s)
  \end{array}
\end{equation*}
  \caption{Semantics}
  \label{fig:semantics}
\end{figure}

\begin{figure}[tbp]
  \framebox{$Γ|-_v t : τ$}
  \begin{mathpar}
  { \inferrule*[Left=Var] {Γ(x) = ∀\alist{χ}.τ} {Γ|-_v x~[\alist{κ}] :
  τ[\alist{χ}|->\alist{κ}]} } \and
  { \inferrule*[Left=PureFun] {Γ,x : τ' |-_v t : τ} {Γ|-_v λx : τ'.t : τ' ->
  τ} } \and
  { \inferrule*[Left=Rec] 
  { Γ' = Γ, x : τ' \\ Γ'' = Γ',~f : τ' ->^ε τ \\ Γ'' |- t : τ,ε \\ Γ' |-_l
  p : \efft{ε} -> \propml \\ Γ' |-_l q : \efft{ε} ->
  \efft{ε} -> τ -> \propml } 
      {Γ|-_v \recml~f~(x : τ').\{p\}t\{q\} : τ' ->^ε τ} 
  } \and
  { \inferrule*[Left=PureApp]
    {Γ |-_v t1 : τ' -> τ \\ Γ|-_v t2 : τ'}
    {Γ |-_v  t1 ~ t2 : τ}
  } \and
  { \inferrule*[Left=Ref]
    { }
    { Γ |- r : Γ(r)~\refml_r }
  }
  \end{mathpar}
  \caption{Value typing}
  \label{fig:valuetyping}
\end{figure}

\paragraph{Value typing.}

\begin{figure}[tpb]
  \framebox{$Γ|- t : τ,ε$}
\begin{mathpar}
  { \inferrule*[Left=Value]
  { Γ |-_v t : τ  }
  { Γ |- t : τ,\emptyset  }
  } \and
  { \inferrule*[Left=App]
    {Γ |-_v t1 : τ' ->^ε τ \\ Γ|-_v t2 : τ'}
    {Γ |- t1 ~ t2 : τ, ε}
  } \and
  { \inferrule*[Left=ite]
    { Γ |- _v t1 : \boolml \\ Γ |- t2 : τ, ε1 \\ Γ |- t3 : τ, ε2 }
    { Γ |- \ifml~ t1 ~\thenml~ t2 ~\elseml ~ t3 : τ, ε1  ε2 }
  } \and
  { \inferrule*[Left=Letv]
    { Γ, \alist{χ} |-_v t1 : τ' \\ Γ, x : ∀\alist{χ}.τ' |- t2 : τ,ε }
    { Γ |- \letml~ x~[\alist{χ}] = t1 ~ \inml~ t2 : τ, ε}
  } \and
  { \inferrule*[Left=Let]
    { Γ |- t1 : τ', ε \\ Γ, x : τ' |- t2 : τ, ε }
    { Γ |- \letml~ x = t1 ~ \inml~ t2 : τ, ε }
  } \and
  { \inferrule*[Left=Sub]
    { Γ |- t : τ', ε \\ ε \subseteq ε' }
    { Γ |- t : τ, ε' }
  }
\end{mathpar}
  \caption{Program typing}
  \label{fig:progtyping}
\end{figure}

\paragraph{Program typing.}

\begin{figure}[tpb]
  \begin{eqnarray*}
    \ceil{α} &=& α\\
    \ceil{τ~\refml_ρ} &=& \ceil{τ}~\refml_ρ\\
    \ceil{τ -> τ'} &=& \ceil{τ} -> \ceil{τ'}\\
    \ceil{τ \times τ'} &=& \ceil{τ} \times \ceil{τ'}\\
    \ceil{\efft{ε}} &=& \efft{ε}\\
    \ceil{τ ->^ε τ'} &=& (\ceil{τ} -> \efft{ε} -> \propml)\times (\ceil{τ} ->
    \efft{ε} -> \efft{ε} -> \ceil{τ'} -> \propml)
  \end{eqnarray*}
  \caption{Lifting types}
  \label{fig:typelift}
\end{figure}

\begin{figure}[tbp]
  \framebox{$Γ|-_l t : τ$}
  \begin{mathpar}
  { \inferrule*[Left=Logic]
    { Γ |-_v t : τ  }
    { Γ |-_l t : \ceil{τ} }
  } \qquad \and
  { \inferrule*[Left=LetLogic]
    { Γ, \alist{χ} |-_l : t' : τ' \\ Γ, x: ∀\alist{χ}.τ' |-_l t : τ  }
    { Γ |-_l \letml~x~[\alist{χ}] = t'~\inml~t : τ }
  }\and
  { \inferrule*[Left=IteLogic]
    { Γ |- _l t1 : \boolml \\ Γ |-_l t2 : τ \\ Γ |-_l t3 : τ }
    { Γ |- \ifml~ t1 ~\thenml~ t2 ~\elseml ~ t3 : τ }
  }
  \end{mathpar}
  \caption{Logic typing}
  \label{fig:logictyping}
\end{figure}

\begin{figure}[tbp]
  \begin{eqnarray*}
    \ceil{x~[\alist{κ}]} &=& x~[\alist{κ}]\\
    \ceil{λx:τ.t} &=& λx:τ.\ceil{t}\\
    \ceil{\recml~f~(x:τ).\{p\} t \{q\}} &=& 
    (λx:\ceil{τ}.p,λx:\ceil{τ}.q)\\
    \ceil{t_{τ' -> τ}~t'} &=& \ceil{t}~\ceil{t'}
  \end{eqnarray*}
  \caption{Lifting values}
  \label{fig:valuelift}
\end{figure}

\begin{figure}[tbp]
  \begin{eqnarray*}
    c(r) &=& \Trueml \\
    c(x~[\alist{κ}]) &=& \Trueml \\
    c(λx:τ.t) &=& ∀x:\ceil{τ}.c(t) \\
    c(t~t') &=& c(t) /\ c(t') \\
    c(\recml~f~(x:τ).\{p\}(t : τ',ε)\{q\}) &=&
    ∀x:\ceil{τ}.∀f:\ceil{τ->^ετ'}.  f = \ceil{v} => \\& &
    ∀s:\efft{ε}.p~s => \wpre_s(t,q~s) \\
    \wpre_s(v, q) &=& c(v) /\ q~s~\ceil{v}\\
    \wpre_s( t_{τ'->^ε τ} t' ,q) &=& \correct{t} /\ \correct{t'} /\
    fst~\ceil{t}~\ceil{t'}~s /\ \\ 
    & & ∀s':\efft{ε}∀ x:\ceil{τ}.~  snd~\ceil{t}~\ceil{t'}~s~s'~x => q~s'~x\\
    \wpre_s(\letml~x~[\alist{χ}] = v~\inml~t, q) &=&
      ∀\alist{χ}.c(v) /\ \letml~x~[\alist{χ}]=\ceil{v}~\inml~ \wpre(t,q)\\
    \wpre_s(\letml~x = ( t1 : τ,ε) ~\inml~ t2 , q) &=&
    \wpre_s( t1 ,λs':\efft{ ε }.λx:\ceil{τ}.\wpre_{s'}( t2 , q ))\\
    \wpre_s(\ifml~ t1 ~\thenml~ t2 ~\elseml~ t3 , q) &=&
      \ifml~ \ceil{ t1 }~\thenml~ \wpre_s( t2 , q)~\elseml~\wpre_s( t3 , q) \\
      \wpre_s( (t : ε <: ε'), q) &=& \wpre_{s|_ε}(t, λs':\efft{ε}.q~(s\oplus s'))
  \end{eqnarray*}
  \caption{The weakest precondition calculus}
  \label{fig:wp}
\end{figure}

\begin{definition}
  We define $s||-_ε f(s)$ to be $|= (λs:\efft{ε}. f)~s|_ε$, where $ε$ is an
  effect expression which only contains region variables in the store.
\end{definition}

In the following, we only consider well-typed formulae.

\begin{thm}
  For any well-typed $t$ of effect $ε$, $s ||-_ε \wpre_s(t,q)$ and $s,t --\ s',t'$
  implies $s'||-_ε \wpre_s( (t : ε' <: ε), q)$.
\end{thm}
\begin{proof}
  
\end{proof}

\end{document}
