\documentclass[a4paper]{llncs}

\usepackage[utf8x]{inputenc}
\usepackage[T1]{fontenc} 
\DeclareUnicodeCharacter{1013}{\ensuremath{\epsilon}}
\newcommand{\textalpha}{\ensuremath{\alpha}}
\providecommand{\textbeta}{\ensuremath{\beta}}
\newcommand{\textchi}{\ensuremath{\chi}}
\newcommand{\textdelta}{\ensuremath{\delta}}
\newcommand{\textepsilon}{\ensuremath{\varepsilon}}
\newcommand{\texteta}{\ensuremath{\eta}}
\newcommand{\textiota}{\ensuremath{\iota}}
\newcommand{\textkappa}{\ensuremath{\kappa}}
\newcommand{\textlambda}{\ensuremath{\lambda}}
\providecommand{\textmu}{\ensuremath{\mu}}
\newcommand{\textomega}{\ensuremath{\omega}}
\newcommand{\textpi}{\ensuremath{\pi}}
\newcommand{\textphi}{\ensuremath{\varphi}}
\newcommand{\textrho}{\ensuremath{\rho}}
\newcommand{\textomicron}{\ensuremath{o}}
\newcommand{\textsigma}{\ensuremath{\sigma}}
\newcommand{\texttau}{\ensuremath{\tau}}
\newcommand{\texttheta}{\ensuremath{\theta}}
\newcommand{\textxi}{\ensuremath{\xi}}
\newcommand{\textzeta}{\ensuremath{\zeta}}

\newcommand{\textLambda}{\ensuremath{\Lambda}}
\newcommand{\textDelta}{\ensuremath{\Delta}}
\newcommand{\textGamma}{\ensuremath{\Gamma}}
\newcommand{\textPi}{\ensuremath{\Pi}}
\newcommand{\textPhi}{\ensuremath{\Phi}}
\newcommand{\textSigma}{\ensuremath{\Sigma}}

\usepackage{stmaryrd}
\usepackage{amsmath,amssymb}
\usepackage{mathpartir}

\newcommand{\wpre}{ {\bf wp}}
\newcommand{\sep}{ ~|~ }
\newcommand{\letml}{{\bf let}}
\newcommand{\inml}{{\bf in}}
\newcommand{\ifml}{{\bf if}}
\newcommand{\thenml}{{\bf then}}
\newcommand{\elseml}{{\bf else}}
\newcommand{\refml}{{\bf ref}}
\newcommand{\propml}{{\bf prop}}
\newcommand{\boolml}{{\bf bool}}
\newcommand{\recml}{{\bf rec}}
\newcommand{\efft}[1]{ \langle #1 \rangle }
\newcommand{\alist}[1]{ \overline{#1} }

\begin{document}
\title{\mbox{Specifying Imperative Higher Order Programs}}

\author{Johannes Kanig \and Jean-Christophe Fill\^atre}
\institute{
Laboratoire de Recherche en Informatique, Univ Paris-Sud, CNRS, Orsay F-91405\\
INRIA Futurs, ProVal, Parc Orsay Universit\'e, F-91893
}
%TODO affiliation

\maketitle

\begin{abstract}
  %TODO
\end{abstract}


\section{Introduction.}
%TODO Hoare logic = sharing between logic and prog language + alias free

\section{A programming language for specification.}

\subsection{Syntax}
\begin{figure}[htpb]
\begin{eqnarray*}
  x & & \text{Program Variables}\\
  α & & \text{Type Variables}\\
  e & & \text{Effect Variables}\\
  ρ & & \text{Region Variables}\\
  χ &::=& α \sep e \sep ρ \\
  t,p,q &::=& x~[\alist{κ}] \sep λx:τ.t \sep \recml~f~(x:τ).\{p\}t\{q\} \sep t~t\sep \\ 
  & & \letml~ x~[\alist{χ}] = t~\inml~t \sep \ifml~t~\thenml~t~\elseml~t\\
  τ &::=& α \sep τ -> τ \sep τ ->^ε τ \sep τ \times τ \sep τ~\refml_ρ \sep
  \efft{ε} \\
  ε &::=& (ρ \sep e)*\\
  κ &::=& τ \sep ε \sep ρ
\end{eqnarray*}
  \caption{Syntax}
  \label{fig:syntax}
\end{figure}

Our language is basically λ-calculus with \letml-bindings. For technical
reasons, the presentation is different from its usual form.
Fig.\ref{fig:syntax} describes the whole language.

Let us first look at the {\em types}. Type variables are of course types, as
well as tuples ($τ\times τ$) and pure functions ($τ -> τ$). There is a second
function type $τ->^ε τ$ for impure functions, annotated with an {\em effect}
$ε$. Effects are simply lists of {\em regions} $ρ$ and {\em effect variables}
$e$. The type expression $τ~\refml_ρ$ describes the type of references in
region $ρ$ containing objects of type $τ$. The type $\efft{ε}$ describes the
portion of the store represented by the effect expression $ε$.

The syntax of terms contains some differences to standard λ-calculus. The
construct $λx:τ.t$ serves to construct a pure function, while the construct
$\recml~f~(x:τ).\{p\}t\{q\}$ can be used to build impure, possibly recursive
functions. The terms $p$ and $q$ are the {\em precondition} and the {\em
postcondition} of the so defined anonymous function. This is of course
reminiscent of the syntax of Hoare Logic. 

Type, region and effect variables must be explicitly generalized and
instantiated. For this purpose, we have defined the metavariable $χ$,
representing variables, and $κ$, representing the corresponding instantiation
objects types, regions and effects\footnote{We do not distinguish between
region variables and region constants; a region constant is simply a region
variable whose scope is the entire program.}. Now, it can be seen that
generalization takes place at \letml-bindings and instantiation at variable
occurrences.
%TODO describe the overline

%TODO : get rid of instantiations ??
\subsection{Typing}

\begin{figure}[htpb]
\begin{mathpar}
  { \inferrule*[Left=Var] {Γ(x) = ∀\alist{χ}.τ} {Γ|-_v x~\alist{κ} :
  τ[\alist{χ}|->\alist{κ}]} } \and
  { \inferrule*[Left=PureFun] {Γ,x : τ' |-_v t : τ} {Γ|-_v λx : τ'.t : τ' ->
  τ} } \and
  { \inferrule*[Left=Rec] 
  { Γ' = Γ,~f : τ' ->^ε τ,~ x : τ' \\ Γ' |- t : τ,ε \\ Γ', cur : \efft{ε} |- p : \propml \\ 
  Γ', old, cur : \efft{ε} |- q : τ -> \propml } 
      {Γ|-_v \recml~f~(x : τ').\{p\}t\{q\} : τ' ->^ε τ} 
  } \and
  { \inferrule*[Left=PureApp]
    {Γ |-_v : t1 : τ' -> τ \\ Γ|-_v : t2 : τ'}
    {Γ |-_v : t1 ~ t2 : τ}
  } \and
  { \inferrule*[Left=App]
    {Γ |-_v : t1 : τ' ->^ε τ \\ Γ|-_v : t2 : τ'}
    {Γ |- : t1 ~ t2 : τ, ε}
  } \and
  { \inferrule*[Left=ite]
    { Γ |- _v : t1 : \boolml \\ Γ |- t2 : τ, ε1 \\ Γ |-t3 : τ, ε2 }
    { Γ |- \ifml~ t1 ~\thenml~ t2 ~\elseml ~ t3 : τ, ε1  ε2 }
  } \and
  { \inferrule*[Left=Letv]
    { Γ, \alist{χ} |-_v : t1 : τ' \\ Γ, x : ∀\alist{χ}.τ' |- t2 : τ,ε }
    { Γ |- \letml~ x~[\alist{χ}] = t1 ~ \inml~ t2 : τ, ε}
  } \and
  { \inferrule*[Left=Let]
    { Γ |- : t1 : τ', ε1 \\ Γ, x : τ' |- t2 : τ, ε2 }
    { Γ |- \letml~ x = t1 ~ \inml~ t2 : τ, ε1  ε2 }
  }
\end{mathpar}
  \caption{The Typing relation}
  \label{fig:typing}
\end{figure}

Again in order to simplify future definitions, the definition typing relation
differs somewhat from the standard way (see Fig.\ref{fig:typing}). First of
all, we present the language in {\em A-normal form} (ANF), where all values
obtained by intermediate computations are named. However, our definition of
values is not syntactical: our ``semantic'' values are objects which are
either syntactic values (abstractions and variables) or are obtained by {\em
pure application}, the application which involves the pure arrow $->$.
Furthermore, the system is a type and effect system. There are in fact two
typing relations, the first one is of the form $Γ|-_v : t : τ$. In this case,
$t$ is a semantic value of type $τ$ and cannot have any effect. In the other
case, we have $Γ|- : t : τ, ε$, where $t$ is an expression of type $τ$ and
effect $ε$. 

Effects can be produced only by function application.


predefined : \propml \boolml


unusual : 
* ANF 
* values defined by types
* two different types of arrows
* distinction bool and prop
* notion of substitution - correspondence χ κ
* TODO logic typing

missing : 
* value injection
* subeffecting

\section{A Simple Weakest Preconditions Calculus.}

\section{Extensions.}

subtyping: effects

\section{Implementation.}

\section{Examples.}

\section{Related Work.}

\section{Conclusion.}

\end{document}
