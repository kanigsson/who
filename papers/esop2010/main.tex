\documentclass[a4paper]{llncs}

\usepackage[utf8x]{inputenc}
\usepackage[T1]{fontenc} 
\DeclareUnicodeCharacter{1013}{\ensuremath{\epsilon}}
\newcommand{\textalpha}{\ensuremath{\alpha}}
\providecommand{\textbeta}{\ensuremath{\beta}}
\newcommand{\textchi}{\ensuremath{\chi}}
\newcommand{\textdelta}{\ensuremath{\delta}}
\newcommand{\textepsilon}{\ensuremath{\varepsilon}}
\newcommand{\texteta}{\ensuremath{\eta}}
\newcommand{\textiota}{\ensuremath{\iota}}
\newcommand{\textkappa}{\ensuremath{\kappa}}
\newcommand{\textlambda}{\ensuremath{\lambda}}
\providecommand{\textmu}{\ensuremath{\mu}}
\newcommand{\textomega}{\ensuremath{\omega}}
\newcommand{\textpi}{\ensuremath{\pi}}
\newcommand{\textphi}{\ensuremath{\varphi}}
\newcommand{\textrho}{\ensuremath{\rho}}
\newcommand{\textsigma}{\ensuremath{\sigma}}
\newcommand{\texttau}{\ensuremath{\tau}}
\newcommand{\texttheta}{\ensuremath{\theta}}
\newcommand{\textxi}{\ensuremath{\xi}}
\newcommand{\textzeta}{\ensuremath{\zeta}}

\newcommand{\textLambda}{\ensuremath{\Lambda}}
\newcommand{\textDelta}{\ensuremath{\Delta}}
\newcommand{\textGamma}{\ensuremath{\Gamma}}
\newcommand{\textPi}{\ensuremath{\Pi}}
\newcommand{\textPhi}{\ensuremath{\Phi}}
\newcommand{\textSigma}{\ensuremath{\Sigma}}

\usepackage{stmaryrd}
\usepackage{amsmath,amssymb}
\usepackage{mathpartir}

\newcommand{\wpre}{ {\bf wp}}
\newcommand{\sep}{ ~|~ }
\newcommand{\letml}{{\bf let}}
\newcommand{\inml}{{\bf in}}
\newcommand{\ifml}{{\bf if}}
\newcommand{\thenml}{{\bf then}}
\newcommand{\elseml}{{\bf else}}
\newcommand{\refml}{{\bf ref}}
\newcommand{\propml}{{\bf prop}}
\newcommand{\boolml}{{\bf bool}}
\newcommand{\recml}{{\bf rec}}
\newcommand{\efft}[1]{ \langle #1 \rangle }
\newcommand{\lift}[1]{ \langle #1 \rangle }
\newcommand{\alist}[1]{ \overline{#1} }
\newcommand{\Trueml}{{\bf True}}
\newcommand{\Falseml}{{\bf False}}
\newcommand{\trueml}{{\bf true}}
\newcommand{\falseml}{{\bf false}}
\newcommand{\unitml}{{\bf unit}}
\newcommand{\coq}{Coq}
\newcommand{\who}{Who}
\newcommand{\correct}[1]{ c(#1) }

\newcommand{\ceil}[1]{\lceil #1 \rceil}

\begin{document}
\title{\mbox{Specifying Imperative Higher Order Programs}}

\author{Johannes Kanig \and Jean-Christophe Fill\^atre}
\institute{
Laboratoire de Recherche en Informatique, Univ Paris-Sud, CNRS, Orsay F-91405\\
INRIA Futurs, ProVal, Parc Orsay Universit\'e, F-91893
}
%TODO affiliation
%TODO find notions for:
%  * ``semantic values'' v
%  * ``store types'' <ε>
% 


\maketitle

\begin{abstract}
  %TODO
\end{abstract}


\section{Introduction}
%TODO Hoare logic = sharing between logic and prog language + alias free

also one or two examples in the described language

characteristics : 
\begin{itemize}
  \item Hoare logic
  \item logic and program share values  
  \item Type and effect system
  \item Region typing to track references
  \item the wp needs absence of aliasing to produce good POs
\end{itemize}

\section{A programming language for specification}

technical presentation:
\begin{itemize}
  \item ANF
  \item values defined by typing 
  \item  two different types of arrows
  \item  distinction bool and prop
  \item  notion of substitution - correspondence χ κ
  \item semantics is standard (goes to appendix )
\end{itemize}

\subsection{Syntax}
\begin{figure}[htpb]
\begin{eqnarray*}
  x & & \text{Program Variables}\\
  α & & \text{Type Variables}\\
  e & & \text{Effect Variables}\\
  ρ & & \text{Region Variables}\\
  χ &::=& α \sep e \sep ρ \\
  t,p,q &::=& x~[\alist{κ}] \sep λx:τ.t \sep \recml~f~(x:τ).\{p\}t\{q\} \sep t~t\sep \\ 
  & & \letml~ x~[\alist{χ}] = t~\inml~t \sep \ifml~t~\thenml~t~\elseml~t\\
  τ &::=& α \sep τ -> τ \sep τ ->^ε τ \sep τ \times τ \sep τ~\refml_ρ \sep
  \efft{ε} \\
  ε &::=& (ρ \sep e)*\\
  κ &::=& τ \sep ε \sep ρ
\end{eqnarray*}
  \caption{Syntax}
  \label{fig:syntax}
\end{figure}

Our programming language (Fig.~\ref{fig:syntax}) is basically the λ-calculus
with \letml-bindings, where there are two different λ-bindings. The first one
($λx:τ.t$) describes a pure, non-recursive binding, and the second one
($\recml~f~(x:τ).\{p\}t\{q\}$) describes recursive, potentially impure
functions; $p$ and $q$ in this term are {\em specifications}, which are
of the same syntax as programs, but are differently typed (typing will be
discussed in the next section). The term $p$ corresponds to the {\em
precondition}, $q$ to the {\em postcondition} of the function. The analogy to
Hoare triples is of course obvious. As usual, there are also variables,
\letml-bindings and an \ifml-\thenml-\elseml-construct. Variables on the type
level can be generalized at \letml-boundaries and have to be instantiated when
the corresponding program variable is used.

Before we describe the syntax of types, let us first explain {\em regions} and
{\em effects}. Regions are simply type level names for memory locations. An
effect is a set of memory locations and {\em effect variables}. An effect
variable stands for an unknown effect. Effects are used to specify
of memory locations an expression can write. They are also used in function
types to describe the memory locations which may be written if the function is
called.

Type, region and effect variables must be explicitly generalized and
instantiated. For this purpose, we have defined the metavariable $χ$,
representing variables, and $κ$, representing the corresponding instantiation
objects types, regions and effects\footnote{We do not distinguish between
region variables and region constants; a region constant is simply a region
variable whose scope is the entire program.}. Now, it can be seen that
generalization takes place at \letml-bindings and instantiation at variable
occurrences. We use a horizontal bar over terms (like in ${\overline{χ}}$) to
denote lists.

\subsection{Typing}

\begin{figure}[htbp]
  \begin{mathpar}
  { \inferrule*[Left=Var] {Γ(x) = ∀\alist{χ}.τ} {Γ|-_v x~\alist{κ} :
  τ[\alist{χ}|->\alist{κ}]} } \and
  { \inferrule*[Left=PureFun] {Γ,x : τ' |-_v t : τ} {Γ|-_v λx : τ'.t : τ' ->
  τ} } \and
  { \inferrule*[Left=Rec] 
  { Γ' = Γ, x : τ' \\ Γ'' = Γ',~f : τ' ->^ε τ \\ Γ'' |- t : τ,ε \\ Γ', cur : \efft{ε} |-_l p : \propml \\ 
  Γ', old, cur : \efft{ε} |-_l q : τ -> \propml } 
      {Γ|-_v \recml~f~(x : τ').\{p\}t\{q\} : τ' ->^ε τ} 
  } \and
  { \inferrule*[Left=PureApp]
    {Γ |-_v t1 : τ' -> τ \\ Γ|-_v t2 : τ'}
    {Γ |-_v  t1 ~ t2 : τ}
  }
  \end{mathpar}
  \caption{Value typing}
  \label{fig:valuetyping}
\end{figure}

\begin{figure}[htpb]
\begin{mathpar}
  { \inferrule*[Left=Value]
  { Γ |-_v t : τ  }
  { Γ |- t : τ,\emptyset  }
  } \and
  { \inferrule*[Left=App]
    {Γ |-_v t1 : τ' ->^ε τ \\ Γ|-_v t2 : τ'}
    {Γ |- t1 ~ t2 : τ, ε}
  } \and
  { \inferrule*[Left=ite]
    { Γ |- _v t1 : \boolml \\ Γ |- t2 : τ, ε1 \\ Γ |- t3 : τ, ε2 }
    { Γ |- \ifml~ t1 ~\thenml~ t2 ~\elseml ~ t3 : τ, ε1  ε2 }
  } \and
  { \inferrule*[Left=Letv]
    { Γ, \alist{χ} |-_v t1 : τ' \\ Γ, x : ∀\alist{χ}.τ' |- t2 : τ,ε }
    { Γ |- \letml~ x~[\alist{χ}] = t1 ~ \inml~ t2 : τ, ε}
  } \and
  { \inferrule*[Left=Let]
    { Γ |- t1 : τ', ε1 \\ Γ, x : τ' |- t2 : τ, ε2 }
    { Γ |- \letml~ x = t1 ~ \inml~ t2 : τ, ε1  ε2 }
  }
\end{mathpar}
  \caption{Program typing}
  \label{fig:progtyping}
\end{figure}

\begin{figure}[htbp]
  \begin{mathpar}
  { \inferrule*[Left=Logic]
    { Γ |-_v t : τ  }
    { Γ |-_l t : \ceil{τ} }
  } \and
  { \inferrule*[Left=LetLogic]
    { Γ, \alist{χ} |-_l : t' : τ' \\ Γ, x: ∀\alist{χ}.τ' |- t : τ  }
    { Γ |-_l \letml~x~[\alist{χ}] = t'~\inml~t : τ }
  }
  \end{mathpar}
  \caption{Logic typing}
  \label{fig:logictyping}
\end{figure}

\begin{figure}[htpb]
  \begin{eqnarray*}
    \ceil{α} &=& α\\
    \ceil{τ~\refml_ρ} &=& \ceil{τ}\refml_ρ\\
    \ceil{τ -> τ'} &=& \ceil{τ} -> \ceil{τ'}\\
    \ceil{τ \times τ'} &=& \ceil{τ} \times-> \ceil{τ'}\\
    \ceil{\efft{ε}} &=& \efft{ε}\\
    \ceil{τ ->^ε τ'} &=& (\ceil{τ} -> \efft{ε} -> \propml)\times (\ceil{τ} ->
    \efft{ε} -> \efft{ε} -> \ceil{τ'} -> \propml)()
  \end{eqnarray*}
  \caption{Lifting types}
  \label{fig:typelift}
\end{figure}

\begin{figure}[htbp]
  \begin{eqnarray*}
    \ceil{x~[\alist{κ}]} &=& x~[\alist{κ}]\\
    \ceil{λx:τ.t} &=& λx:τ.\ceil{t}\\
    \ceil{\recml~f~(x:τ).\{p\} (t : ε ) \{q\}} &=& 
    (λx:\ceil{τ}.λcur:\efft{ε}.p,λx:\ceil{τ}.λold,cur:\efft{ε}.q)\\
    \ceil{t_{τ' -> τ}~t'} &=& \ceil{t}~\ceil{t'}
  \end{eqnarray*}
  \caption{Lifting values}
  \label{fig:valuelift}
\end{figure}

We have already seen that programs and specification share in fact the same
syntax. Though they are differently typed, their typing relations still share
a lot. For this reason, and to underline the fact that in Hoare logic, values
are shared between programs and the logic, we present three different typing
relations. One for programs, one for specifications, and the common part is
factored out in a third typing relation for values.

Again in order to simplify future definitions, the definition typing relation
differs somewhat from the standard way (see Fig.~\ref{fig:progtyping}). First of
all, we present the language in {\em A-normal form} (ANF), where all values
obtained by intermediate computations are named. However, our definition of
values is not syntactical: our ``semantic'' values are objects which are
either syntactic values (abstractions and variables) or are obtained by {\em
pure application}, the application which involves the pure arrow $τ->τ$.
Furthermore, the system is a type and effect system. There are in fact two
typing relations, the first one is of the form $Γ|-_v  t : τ$. In this case,
$t$ is a semantic value of type $τ$ and cannot have any effect. It can be seen
that the set of terms described by $|-_v$ is precisely the one obtained by
combining syntactic values and pure application in a well-typed way.

In the other case, we have $Γ|- t : τ, ε$, where $t$ is an expression of
type $τ$ and effect $ε$. Impure applications, \letml-expressions and
\ifml-expressions can only be typed in this way. Effects can be produced only
by function application, where the impure arrow $τ'->^ετ$ is involved. In this
case, the effect of the application is precisely the effect $ε$ of the arrow,
because the two subterms are semantic values and don't have any effect.

The typing relies on two predefined types, \propml\ and \boolml. \propml\ is
the logical truth and contains the values \Trueml\ and \Falseml, while
\boolml\ describes boolean (decidable) properties and contains \trueml\ and
\falseml. The distinction between \propml\ and \boolml\ is the same one as in
\coq, for example. Properties expressed in \propml\ are potentially
undecidable and one cannot take decisions by testing their truth value. On the
other hand, \boolml\ talks about decidable properties and so it makes sense to
expect the predicate of the \ifml-statement to be of type \boolml.

missing : 
* subeffecting

%TODO is the LOGIC rule correct ??

\subsection{predefined constants}

arithmetic operations, operations on pairs, logic operations on prop, pre/post

reference operations
!!

operations on maps : restrict, combine
\subsection{Sharing between logic and programs}

use logic application in programs : no problem;
use lambdas with effects in logic : specification

\subsection{References}

We haven't mentioned references yet, but all we need to support them is to add
the three usual functions $ref, :=, !$ for reference creation, assignment and
lookup, respectively, to the environment. In our system, these functions have
the following types:
\begin{eqnarray*}
  ref &:& ∀αρ.α ->^ρ α~\refml_ρ\\
  := &:& ∀αρ.α~\refml_ρ ->^\emptyset α ->^ρ \unitml\\
  ! &:& ∀αρ. α~\refml_ρ ->^ρ α
\end{eqnarray*}

\subsection{Alias restriction}

\begin{figure}[htpb]
  \begin{mathpar}
    {\inferrule*[Left=Var]
      {Γ(x) = ∀\alist{χ}.τ \\ σ = [\alist{χ}|->\alist{κ}] \\ σ \sim τ }
      {Γ |- x : τσ }
    } \\
    { \inferrule*[Left=LetRef, Right={ρ \textup{fresh}, $ρ\notin τ$}]
      {Γ |-_v t1 : τ' \\ Γ, x : τ'~\refml_ρ |- t2 : τ, ε }
      {Γ |- \letml~x = ref~ t1 ~\inml~ t2 : τ, ε \setminus ρ}
    }
  \end{mathpar}
  \caption{A modified and a new rule for aliasing}
  \label{fig:aliasing}
\end{figure}

\subsection{Some properties}

$Γ|-_v v : τ$ implies $Γ |-_l \ceil{v} : \ceil{τ}$

\section{A Simple Weakest Preconditions Calculus}

In the following, we use the metasymbol $v$ to denote well-typed semantic
values, i.e. terms with a typing derivation $Γ|-_v v : τ$. 

\begin{figure}[htbp]
  \begin{eqnarray*}
    \wpre_s(v, q) &=& q~s~\ceil{v}\\
    \wpre_s( t_{τ'->^ε τ} t' ,q) &=& \correct{t} /\ \correct(t') /\
    fst~\ceil{t}~\ceil{t'}~s /\ \\ 
    & & ∀s':\efft{ε}, x:τ.~  snd~\ceil{t}~\ceil{t'}~s~s'~x => q~s'~x\\
    \wpre_s(\letml~x~[\alist{χ}] = v~\inml~t, q) &=&
      c(v) /\ \letml~x~[\alist{χ}]=\ceil{v}~\inml~ \wpre(t,q)\\
    \wpre_s(\letml~x = ( t1 : ε1 ) ~\inml~ ( t2 : ε2 ) , q) &=&
    \wpre_s( t1 ,λs':\efft{ ε1  ε2 }.\wpre_{s'}(t2, q ))\\
    \wpre_s(\ifml~ t1 ~\thenml~ t2 ~\elseml~ t3 , q) &=&
    (\ceil{ t1 } = \trueml => \wpre_s( t2 , q)) /\ \\
    & &(\ceil{ t1 } = \falseml => \wpre_s( t3 , q)) 
  \end{eqnarray*}
  \caption{The weakest precondition calculus}
  \label{fig:wp}
\end{figure}

\section{Extensions}

subtyping: effects

\section{Implementation}

%what has been implemented

A prototype implementation by the name ``\who'' exists and is freely available. It implements the
described language with type, region and effect polymorphism, higher-order
logics and store types in the logic. It takes as an input a fully annotated
program and outputs proof obligations which have to be proved to guarantee the
correctness of the program with respect to its specification. Currently, only
output to the interactive proof assistant Coq has been implemented. Our logic
is a subset of Coq's, so the translation is particularly simple.

%details
% lines of code : less than 3000 lines of ocaml
% limited form of type and effect inference
% 

\section{Examples}

\begin{center}
    \begin{tabular}{ | r | r | r | r |}
    \hline
    Program & Code & Spec & Proofs \\ \hline
    Array &  &  &  \\
    Hashtbl & & &  \\ \hline
    Memo/Ymemo & & &  \\ \hline
    Koda-Ruskey~\cite{KanigFilliatre09wml} & & &  \\ \hline
    \end{tabular}
\end{center}

\subsection{memo}
\subsection{ymemo}

\section{Related Work}

\section{Conclusion}

\paragraph{Future Work.}

We are currently investigating the possibility to obtain first-order proof
obligations, to be able to use SMT solvers~\cite{RanTin-SMTLIB}, which would
avoid manual proofs in many cases. An obvious solution would be to reuse
existing encodings of higher-order logic in first-order logic, such as the one
by Pottier and Gauthier~\cite{pottier-gauthier-hosc}. This technique has been
successfully employed by the Pangolin system~\cite{regis-gianas-pottier-08}.
Another option comes from our observation that in many cases, the proof
obligations for higher-order functions can actually be expressed in
first-order logic, when one replaces top-level {\em quantifications} of
functional arguments and invariants by top-level {\em declarations}. Doing
this translation manually, we have obtained first-order proof
obligations for the {\tt Array.iter} function which are all discharged by
SMT solvers. We don't know if such a transformation can always be carried out.

what about correspondence between $τ ->^\emptyset τ$ and $τ -> τ$?

\bibliographystyle{plain}
\bibliography{biblio}{}

\end{document}
