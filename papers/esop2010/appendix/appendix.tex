%; whizzy document
\documentclass[a4paper]{article}
\usepackage[utf8x]{inputenc}
\usepackage[T1]{fontenc}
\usepackage{amsmath,amssymb,amsthm}
\usepackage{stmaryrd} % for semantic brackets
\usepackage{mathpartir} % for inference rules
\DeclareUnicodeCharacter{1013}{\ensuremath{\epsilon}}
\newcommand{\textalpha}{\ensuremath{\alpha}}
\providecommand{\textbeta}{\ensuremath{\beta}}
\newcommand{\textchi}{\ensuremath{\chi}}
\newcommand{\textdelta}{\ensuremath{\delta}}
\newcommand{\textepsilon}{\ensuremath{\varepsilon}}
\newcommand{\texteta}{\ensuremath{\eta}}
\newcommand{\textiota}{\ensuremath{\iota}}
\newcommand{\textkappa}{\ensuremath{\kappa}}
\newcommand{\textlambda}{\ensuremath{\lambda}}
\providecommand{\textmu}{\ensuremath{\mu}}
\newcommand{\textomega}{\ensuremath{\omega}}
\newcommand{\textpi}{\ensuremath{\pi}}
\newcommand{\textphi}{\ensuremath{\varphi}}
\newcommand{\textrho}{\ensuremath{\rho}}
\newcommand{\textsigma}{\ensuremath{\sigma}}
\newcommand{\texttau}{\ensuremath{\tau}}
\newcommand{\texttheta}{\ensuremath{\theta}}
\newcommand{\textxi}{\ensuremath{\xi}}
\newcommand{\textzeta}{\ensuremath{\zeta}}

\newcommand{\textLambda}{\ensuremath{\Lambda}}
\newcommand{\textDelta}{\ensuremath{\Delta}}
\newcommand{\textGamma}{\ensuremath{\Gamma}}
\newcommand{\textPi}{\ensuremath{\Pi}}
\newcommand{\textPhi}{\ensuremath{\Phi}}
\newcommand{\textSigma}{\ensuremath{\Sigma}}

\newtheorem{thm}{Theorem}[section]
\newtheorem{cor}[thm]{Corollary}
\newtheorem{lem}[thm]{Lemma}
\newtheorem{prop}[thm]{Proposition}
\theoremstyle{remark}
\newtheorem{rem}[thm]{Remark}
%\theoremstyle{remark}
%\newtheorem{example}[thm]{Example}

\theoremstyle{definition}
\newtheorem{definition}[thm]{Definition}

\newcommand{\myexp}[2]{\ensuremath{#1^{#2}}}
\newcommand{\myindex}[2]{\ensuremath{#1_{#2}}}

\newcommand{\ceil}[1]{\lceil #1 \rceil}
\newcommand{\floor}[1]{\lfloor #1 \rfloor}
\newcommand{\alist}[1]{\overline{#1}}
\newcommand{\ev}{\epsilon}
\newcommand{\sep}{~\mid~}
\newcommand{\ocaml}{OCaml}
\newcommand{\lbracket}{[}
\newcommand{\rbracket}{]}



\author{Johannes Kanig}
\title{Technical Appendix}
\date{\today}
%TODO describe label substitutions clearly
%TODO describe all kinds of substitutions clearly
%TODO introduce new syntax for records : <>, only labels accessible
%... is it reasonable ?
% traduire vers des n-uplets plutôt que des ->
% utiliser la syntaxe concrète de l'outil
%TODO invent new notation!

\newcommand{\reft}{\texttt{ref}}
\newcommand{\conseff}{\ensuremath{\sqcup}}
\newcommand{\refsub}[1]{\texttt{ref}_{#1}}
\newcommand{\redrel}{\ensuremath{\longrightarrow^\star}}
\newcommand{\effharp}[1]{\ensuremath{\rightharpoonup}}
%\newcommand{\effharp}[1]{\ensuremath{\stackrel{#1}{\rightharpoonup}}}
\newcommand{\efftarr}[1]{\ensuremath{\stackrel{#1}{\twoheadrightarrow}}}
\newcommand{\typearr}[1]{\ensuremath{\stackrel{#1}{\rightarrow}}}
\newcommand{\mllet}[3]{\ensuremath{\texttt{let}\ #1=#2\ \texttt{in}\ #3}}
%\newcommand{\typerel}[4]{\ensuremath{#1 \vdash #2 : #3 , #4}}
\newcommand{\supeff}{\ensuremath{\sqsupseteq}}
\newcommand{\subeff}{\ensuremath{\sqsubseteq}}
\newcommand{\letvrule}{\textsc{Let}\ensuremath{_v}}

% \alpha

\newcommand{\bracket}[1]{\ensuremath{\left[ #1 \right]}}
\newcommand{\regpr}[2]{\ensuremath{ \left\{ #1 \right\}.#2 }}
\newcommand{\napp}[2]{\ensuremath{#1\ #2}}
\newcommand{\tyapp}[2]{\ensuremath{#1\ \left[ #2 \right]}}
\newcommand{\rvapp}[2]{\ensuremath{#1\ \left[ #2 \right]}}
\newcommand{\rapp}[2]{\ensuremath{#1\left\{ #2 \right\}}}
\newcommand{\effapp}[2]{\ensuremath{#1\left[ #2 \right]}}
\newcommand{\letreg}[2]{\ensuremath{\texttt{letregion}\ #1\ \texttt{in}\ #2}}
\newcommand{\region}[2]{\ensuremath{\texttt{region}\ #1\ \texttt{in}\ #2}}
\newcommand{\tyarr}{\ensuremath{\rightarrow}}
\newcommand{\refreg}[2]{ {\normalfont \ensuremath{#1\ \refsub{#2}}}}
\newcommand{\regty}[2]{\ensuremath{ \left\{ #1 \right\}.#2 }}
\newcommand{\basered}{\ensuremath{\rightharpoonup }}
\newcommand{\onestep}{\ensuremath{\rightarrow }}
\newcommand{\reduction}{\ensuremath{\twoheadrightarrow }}
\newcommand{\esub}[2]{\ensuremath{#1\left[ #2 \right]}}
\newcommand{\tsub}[2]{\ensuremath{#1\left[ #2 \right]}}
\newcommand{\rvsub}[2]{\ensuremath{#1\left[ #2 \right]}}
\newcommand{\statesub}[2]{\ensuremath{#1\left[ #2 \right]}}
\newcommand{\regsub}[2]{\ensuremath{#1\left[ #2 \right]}}
\newcommand{\cwf}[1]{{\normalfont \ensuremath{\vdash #1\ \texttt{wf}}}}
\newcommand{\ctwf}[2]{{\normalfont \ensuremath{#1 \vdash #2\ \texttt{wf}}}}
\newcommand{\capreg}[1]{\ensuremath{\left\{ #1 \right\}}}
\newcommand{\typerel}[4]{\ensuremath{#1 \vdash #2 : #3, #4 }}
\newcommand{\storetype}[2]{\ensuremath{#1 \vdash #2 }}

\newcommand{\varrule}{{\normalfont\textsc{Var}}}
\newcommand{\conrule}{{\normalfont\textsc{Con}}}
\newcommand{\absrule}{{\normalfont\textsc{Abs}}}
\newcommand{\absregrule}{{\normalfont\textsc{AbsReg}}}
\newcommand{\absdregrule}{{\normalfont\textsc{AbsDReg}}}
\newcommand{\abstyrule}{{\normalfont\textsc{Abs}\ensuremath{_\alpha}}}
\newcommand{\absgenrule}{{\normalfont\textsc{Abs}\ensuremath{_\chi}}}
\newcommand{\absrrule}{{\normalfont\textsc{Abs}\ensuremath{_\rho}}}
\newcommand{\apprule}{{\normalfont\textsc{App}}}
\newcommand{\appregrule}{{\normalfont\textsc{AppReg}}}
\newcommand{\apptyrule}{{\normalfont\textsc{App}\ensuremath{_\alpha}}}
\newcommand{\appgenrule}{{\normalfont\textsc{App}\ensuremath{_\chi}}}
\newcommand{\apprrule}{{\normalfont\textsc{App}\ensuremath{_\rho}}}
\newcommand{\letregrule}{{\normalfont\textsc{LetReg}}}
\newcommand{\regionrule}{{\normalfont\textsc{Region}}}
\newcommand{\locrule}{{\normalfont\textsc{Loc}}}
\newcommand{\appdrule}{{\normalfont\textsc{AppDReg}}}
\newcommand{\subeffrule}{{\normalfont\textsc{Sub}}}
\newcommand{\recrule}{{\normalfont\textsc{Rec}}}
\newcommand{\letrule}{{\normalfont\textsc{Let}}}

\newcommand{\letref}[3]{\mllet{#1}{\reft\ #2}{#3}}
\newcommand{\effarr}[1]{\ensuremath{\stackrel{#1}{\rightarrow}}}
\newcommand{\baseredwhy}{\ensuremath{\rightharpoonup_w }}
\newcommand{\onestepwhy}{\ensuremath{\rightarrow_w }}
\newcommand{\reductionwhy}{\ensuremath{\twoheadrightarrow_w }}
\newcommand{\wvarrule}{\textsc{W-Var}}
\newcommand{\wconrule}{\textsc{W-Con}}
\newcommand{\wabsrule}{\textsc{W-Abs}}
\newcommand{\wapprule}{\textsc{W-App}}
\newcommand{\wapptyrule}{\textsc{W-App}\ensuremath{_\alpha}}
\newcommand{\wabstyrule}{\textsc{W-Abs}\ensuremath{_\alpha}}
\newcommand{\wapprefrule}{\textsc{W-AppRef}}
\newcommand{\wletrefrule}{\textsc{W-LetRef}}
\newcommand{\wtyperel}[4]{\ensuremath{#1 \vdash #2 : #3 , #4}}

\newcommand{\translate}[2]{\ensuremath{\llbracket #1 \rrbracket_{#2} }}
\newcommand{\tytranslate}[1]{\ensuremath{\llbracket #1 \rrbracket }}


\newcommand{\bmark}[1]{\textbf{(#1)}}
\newcommand{\pre}{{\bf pre}}
\newcommand{\post}{{\bf post}}
\newcommand{\wpre}{{\bf wp}}
\newcommand{\proptype}{{\bf prop}}
\newcommand{\rec}{{\bf rec}}
\newcommand{\void}{{\tt ()}}
\newcommand{\unit}{{\tt unit}}
\newcommand{\inttype}{{\tt int}} 
\newcommand{\letst}{{\bf let }}
\newcommand{\inst}{{\bf in }}
\newcommand{\true}{{\tt true}}
\newcommand{\false}{{\tt false}}
\newcommand{\bool}{{\tt bool}} 
\newcommand{\ifst}{{\bf if }}
\newcommand{\thenst}{{\bf then }}
\newcommand{\elsest}{{\bf else }}
\newcommand{\evlist}{\alist{\ev}}
\newcommand{\False}{\bot}
\newcommand{\True}{\top}
\newcommand{\reftype}{{\bf ref}}

\begin{document}
\maketitle

\section{The Aliasing Restriction}

In this section, we deliver the correctness proof for restriction to singleton
regions. We show here that each region can in fact only contain a single
reference. Instead of the {\bf letref}-construct as in the paper, we develop a
more general approach using a {\bf letreg}-construct. The principal idea is
that to call the function $\mathit{ref}$ to create a reference in a certain
region $ρ$ , one needs a capacity, a special token, denoted $\{ ρ\}$, which is
consumed by the function call. These capacities are treated linearly, i.e.
they cannot be duplicated. The {\bf letreg} construct delimits not only the
scope of a region variable, but also gives a single creation capacity. So the
whole following development basically ensures that no region capacity can be
duplicated, and then we obtain for free that no region contains more than one
memory location. A region can be empty, though.

The {\bf letref}-construct can be trivially recovered of course:
\begin{equation*}
  \letst~x = \mathit{ref}~v~\inst~e \quad \equiv\quad 
   \letreg{ρ}{\mllet{x}{\mathit{ref}~\{ρ\}~v}{e}}
\end{equation*}

\paragraph{A guide to the proof.} After the usual definitions of syntax and
semantics, we define a splitting operation over typing environments which
takes care of the linear nature of creation capacities
(Sec.~\ref{sec:environments}). In Section~\ref{sec:typing} we prove many
auxiliary lemmas for the main theorem, which is stated in
Section~\ref{sec:subjectreduction}. Definition \ref{def:invariants} contains
the important property to be maintained: If a term is well-typed wrt. an
environment, then all regions are singleton regions, and if a creation
capacity on a region is still available, this region is empty. The desired
property follows.

\subsection{Syntax}

\begin{equation*}
  \begin{array}[]{llll}
  \mbox{Program Variables}& x & & \\
  \mbox{Region Variables} & \rho & &\\
  \mbox{Type Variables} & \alpha & &\\
  \mbox{Effect Variables} & \epsilon & &\\
  \mbox{Program Constants} & c & &\\
  \mbox{Values}& v & ::= & c \sep \tyapp{x}{\bar{\tau} \bar{\rho}\bar{\varepsilon}} \sep \lambda x:\tau.e \sep
  \regpr{\rho}{e} \sep y_\rho \\
  \mbox{Expressions} & e & ::= & v\sep  \napp{e}{e} \sep \rapp{e}{\rho} \sep \letreg{\rho}{e} \sep \\
  & & & (e : \varepsilon ) \sep \region{\rho}{e} \sep \mllet{\bar{\alpha} \bar{\rho} \bar{\epsilon}\ x}{v}{e}\\
  \mbox{Types} & \tau & ::= & \alpha \sep \iota \sep \tau \effarr{\varepsilon} \tau \sep  \refreg{\tau}{\rho}
  \sep \regty{\rho,\varepsilon}{\tau} \\ 
  \mbox{Type Schemes} & ο & ::= & \tau\sep  \forall\alpha.\sigma \sep \forall\rho.\sigma \sep \forall\epsilon.\sigma \\
  \mbox{Effects} & \varepsilon & ::= & \epsilon \sep \rho \sep \emptyset \sep \varepsilon \cup \varepsilon
  \end{array}
\end{equation*}

\subsection{Semantics}
\label{sec:semantics}

Before executing a term, effect annotations are erased.

In this overview, when we write $d\in s$ we mean $d\in dom(s)$ and
similar for $l\in s(d)$.
\begin{equation*}
   \begin{array}[t]{rrclr}
    \beta & s, (\lambda x:\tau.e)\ v & \basered & s, \esub{e}{x\mapsto v} & \\
    let & s, \mllet{\bar{\alpha} \bar{\rho} \bar{\epsilon}\ x}{v}{e} & \basered & s,
    \esub{e}{x\mapsto \Lambda\bar{\alpha}\bar{\rho}\bar{\epsilon}.v} \\
    \capreg{\rho} & s, \rapp{(\regpr{\rho}{e})}{\rho} & \basered & s,e & \\
    letreg & s, \letreg{\rho}{e} & \basered & \statesub{s}{\rho_1\mapsto \left\{
    \right\}}, \region{\rho_1}{\regsub{e}{\rho\mapsto \rho_1}} & \rho_1\notin s \\
    region & s, \region{\rho}{v} & \basered & s, v \\
    ref & s, \reft \left[ \xi_1 \dots \xi_n \right] \capreg{\rho_1}\  v &
    \basered & s\left[ \rho_1 \mapsto s(\rho_1)\left[ y\mapsto v \right]\right],
    y_{\rho_1} & \rho_1\in s, y\notin s(\rho_1)\\
    := & s, := \left[ \xi_1 \dots \xi_n \right] y_{\rho_1} v & \basered & s\left[ \rho_1 \mapsto s(\rho_1)\left[
    y \mapsto v \right] \right], \void & \rho_1\in s, y\in s(\rho_1)\\
    ! & s, ! \left[ \xi_1 \dots \xi_n \right] y_{\rho_1} & \basered & s, s(\rho_1)(y) & \rho_1\in s, y\in
    s(\rho_1) \\
    \rec & s, (\rec\ f\ v_1)\ v_2 & \basered & s, v_1\left[ f \mapsto (\rec\
    v_1)  \right] v_2
  \end{array}
\end{equation*}

The relation $\basered$ describes a single reduction step at the top of a
term. To lift this relation to arbitrary expressions, we define call-by-value
contexts:
\begin{equation*}
  E ::= \left[  \right] \sep E\ e \sep v\ E\sep E \left[ \xi \right] \sep
  \rapp{E}{p} \sep \region{\rho}{E} \sep \mllet{x}{E}{e}
\end{equation*}

Now we define the one step reduction relation $\onestep$ as follows:
\begin{mathpar}
  \inferrule* 
    { s, e \basered s', e' }
    { s, E\left[ e \right] \onestep s', E\left[ e' \right] }
\end{mathpar}

The general reduction relation $\reduction$ is defined as the reflexive and
transitive closure of $\onestep$.

\subsection{Environments}
\label{sec:environments}
\begin{eqnarray*}
  \Gamma & ::= & \emptyset \sep \Gamma, x : \tau \sep \Gamma, \chi \sep \Gamma, \capreg{\rho}\\
\end{eqnarray*}

\subsubsection*{Wellformedness}
We write $\chi\notin fv(\Gamma)$ to express that $\chi$ never occurs free in $\Gamma$. We write
$\chi\notin \Gamma$ to express that $\chi$ is not an {\em element} of the environment
$\Gamma$.

\begin{mathpar}
  (1)\ \cwf{\emptyset} \and
  {\inferrule* [left=(2)] {\cwf{\Gamma}\quad \chi\notin \Gamma} {\cwf{\Gamma,\chi}}}\and
  {\inferrule* [left=(3)] {\cwf{\Gamma}\quad \ctwf{\Gamma}{\rho} } 
                          {\cwf{\Gamma,\capreg{\rho}}}}
  \and
  {\inferrule* [left=(4)] {\cwf{\Gamma}\quad \ctwf{\Gamma}{\tau}} {\cwf{\Gamma,x:\tau}}}
  \and
  {\inferrule* [left=(5)] {\chi\in \Gamma}{\ctwf{\Gamma}{\chi}}}\and
  (6)\  \ctwf{\Gamma}{\iota}\and
  {\inferrule* [left=(7)] {\ctwf{\Gamma}{\tau_1}\quad\ctwf{\Gamma}{\tau_2}\quad\ctwf{\Gamma}{\varepsilon}}
  {\ctwf{\Gamma}{\tau_1\effarr{\varepsilon} \tau_2}}}\and
  {\inferrule* [left=(8)] {\ctwf{\Gamma,\bar{\chi}}{\tau}}{\ctwf{\Gamma}{\forall\bar{\chi}.\tau}}}\and
  {\inferrule* [left=(9)] {\ctwf{\Gamma}{\tau} \\ \ctwf{\Gamma}{\rho}}
      {\ctwf{\Gamma}{\refreg{\tau}{\rho}}}} \and
  {\inferrule* [left=(10)] {\ctwf{\Gamma}{\tau} \\ \ctwf{\Gamma}{\rho}}
      {\ctwf{\Gamma}{\regty{\rho}{\tau}}}} \and
  {\inferrule* [left=(11)] {\ctwf{\Gamma}{\varepsilon_1} \\ \ctwf{\Gamma}{\varepsilon_2}}
      {\ctwf{\Gamma}{\varepsilon_1\cup \varepsilon_2}}}
\end{mathpar}

The rules (1) to (4) concern wellformed environment construction, The rules (5) to (10)
concern wellformedness of types. Rule (11) concerns wellformedness of effects
and recycles rule (5) for effect and region variables.

\begin{prop}
  $\chi'\notin \Gamma$ and $\ctwf{\Gamma}{\xi}$ implies $\chi'\notin fv(\xi)$.
  \label{prop:cwffvty}
\end{prop}
\begin{proof}
  Proof by induction on the derivation of $\ctwf{\Gamma}{\xi}$.
  \begin{description}

    \item[Case (5)] If $\chi = \chi'$, then we have a contradiction between the
      premise of the rule and the hypothesis of our claim. In the other case
      we have $\chi'\notin fv(\chi)$.

    \item[Case (6)] There are no free variables in base types. 

    \item[Cases (7), (9), (10), (11)] By induction hypotheses.

    \item[Case (8)] We can assume any $\chi$ of the rule to be different to the
      $\chi'$ in the claim. Now suppose $\chi'\notin \Gamma,\bar{\chi}$ and $\chi'\notin fv(\tau)$. We have
      $\chi'\notin \Gamma$, and $\chi'\notin fv(\tau)$ iff $\chi'\notin fv(\forall\bar{\chi}.\tau)$.
  \end{description}
\end{proof}

\begin{prop}
  \cwf{\Gamma,\chi',\Delta} and \ctwf{\Gamma}{\xi} imply $\chi'\notin fv(\Gamma)$ and
  \cwf{\Gamma,\tsub{\Delta}{\chi'\mapsto \xi}}.
  \label{prop:cwftysub}
\end{prop}
\begin{proof}
  We proceed by induction on the wellformedness derivation of \cwf{\Gamma,\chi',\Delta}. 
  \begin{description}

    \item[Case (1)] The rule states \cwf{\emptyset}, but this cannot be the
      last rule applied to establish \cwf{\Gamma,\chi',\Delta}.

    \item[Case (2)] If $\chi \neq \chi'$ then we can simply conclude by induction
      hypothesis. If $\chi = \chi'$, then either $\Delta=\emptyset$ and the side
      condition $\chi\notin\Gamma$ together with proposition \ref{prop:cwffvty} gives
      exactly the property we want to prove for the first claim, and the
      second claim is trivial since $\tsub{\emptyset}{\chi\mapsto \xi} =
      \emptyset$. Or $\Delta$ is not empty and that side condition cannot be true.

    \item[Case (3)] Either $\chi' = \rho$, in which case $\xi = \rho'$ and we conclude by
      the obvious fact $\ctwf{\Gamma,\tsub{\Delta}{\rho\mapsto \rho'}}{\capreg{\rho'}}$. In the case
      where $\chi' \neq \rho$, we conclude simply by induction hypothesis.

    \item[Case(4)] For the first claim, there is nothing to prove. For the
      second claim, we need to prove that $\ctwf{\Gamma,\chi',\Delta}{\tau}$ implies
      $\ctwf{\Gamma,\tsub{\Delta}{\chi'\mapsto \xi}}{\tsub{\tau}{\chi'\mapsto \xi}}$.

      \begin{description}

        \item[Case (5)] If $\chi = \chi'$, this amounts to proving
          \ctwf{\Gamma,\tsub{\Delta}{\chi\mapsto \chi}}{\xi}, which is obvious because of
          \ctwf{\Gamma}{\xi} by hypothesis.

        \item[Case (6)] This rule is independent of the environment.

        \item[Case (7),(9),(10),(11)] Simply by applying the induction
          hypotheses.

        \item[Case (8)] We can assume $\chi'\notin \bar{\chi}$, and thus conclude by
          induction hypothesis.

      \end{description}
  \end{description}
\end{proof}

\begin{prop}
  \cwf{\Gamma} and $\Delta$ is a prefix of $\Gamma$ implies \cwf{\Delta}.
  \label{prop:cwfprefix}
\end{prop}
\begin{proof}
  We prove the claim by induction over the wellformedness derivation for $\Gamma$. 
\begin{description}
  \item[Case (1)] This implies $\Gamma$ is empty and thus $\Delta$ is empty, so \cwf{\Delta}
    by (1).
  \item[Case (2)] 
    The conclusion of the rule is \cwf{\Gamma,\chi}(a); it's premises are
    $\cwf{\Gamma}$ (b) and $\chi\notin fv(\Gamma)$ (c). If $\Delta = \Gamma,\chi$, then conclude by (a).
    If this is not the case $\Delta$ is also a prefix of $\Gamma$, and by induction
    hypothesis and (b), we can conclude.
  \item[Cases (3) and (4)] completely analogous.
\end{description}
\end{proof}

\begin{prop}
  \label{prop:cwfomit}
  \cwf{\Gamma,x : \tau, \Delta} implies \cwf{\Gamma,\Delta}.
\end{prop}
\begin{proof}
  By induction over the wellformedness derivation for $\Gamma$.
  \begin{description}
    \item[Case (1)] impossible.
    \item[Cases (2), (3), (4) and (5)] By induction hypothesis.  
    \item[Case (6)] By proving that \ctwf{\Gamma,x : \tau, \Delta}{y : \tau'} implies
      \ctwf{\Gamma,\Delta}{y : \tau'}. This is obvious, as one can see that wellformedness
      of types depends only on the presence of type and region variables, and
      not on variable bindings in the environment.
  \end{description}
  
\end{proof}


\subsubsection*{Splitting}

We define a non-deterministic splitting operation on the environment. The
operation is entirely symmetric, so only the ``left''
case of $\oplus_4$ are given.

\begin{equation*}
  \begin{array}[]{lrcl}
  \oplus_1 & \emptyset \oplus \emptyset & = & \emptyset \\
  \oplus_2 & (\Gamma_1, \chi) \oplus (\Gamma_2, \chi) & = & (\Gamma_1 \oplus \Gamma_2),\chi \\
  \oplus_3 & (\Gamma_1, x : \tau) \oplus (\Gamma_2, x : \tau) & = & (\Gamma_1 \oplus \Gamma_2),x:\tau \\
  \oplus_4 & (\Gamma_1,\capreg{\rho}) \oplus \Gamma_2 & = & (\Gamma_1 \oplus
              \Gamma_2),\capreg{\rho}\qquad \capreg{\rho}\notin \Gamma_2 \\
  \end{array}
\end{equation*}

\begin{prop}
  \cwf{\Gamma_1} and \cwf{\Gamma_2} imply \cwf{\Gamma_1\oplus \Gamma_2}, if $\Gamma_1\oplus \Gamma_2$ is
  defined.
  \label{prop:cwfoplus}
\end{prop}
\begin{proof}
  By induction on the construction of $\Gamma_1\oplus \Gamma_2$.
  \begin{description}
    \item[Case $\oplus_1$] Obvious.

    \item[Case $\oplus_2$] The corresponding rule states $(\Gamma_1, \chi) \oplus
      (\Gamma_2, \chi) = (\Gamma_1 \oplus \Gamma_2),\chi$. By the premises and proposition
      \ref{prop:cwfprefix}, we have \cwf{\Gamma_1} and \cwf{\Gamma_2}, and by induction
      hypothesis we have \cwf{\Gamma_1\oplus \Gamma_2}. By the premises of the
      wellformedness derivations for \cwf{\Gamma_1,\chi} and \cwf{\Gamma_2,\chi} we obtain
      $\chi\notin fv(\Gamma_1)$ and $\chi\notin fv(\Gamma_2)$. As can be simply checked, this
      implies $\chi\notin fv(\Gamma_1 \oplus \Gamma_2)$, and by the rule (2) we conclude
      \cwf{(\Gamma_1\oplus \Gamma_2),\chi}.

   \item[Case $\oplus_3$] analogously, we can conclude $\cwf{\Gamma_1}$ and
     \cwf{\Gamma_2} and thus, by induction hypothesis, \cwf{\Gamma_1 \oplus \Gamma_2}. We are
     left to prove that if \ctwf{\Gamma_1}{\tau} then \ctwf{\Gamma_1 \oplus \Gamma_2}{\tau}, which
     is very easily proved by induction. We can conclude \cwf{(\Gamma_1 \oplus
     \Gamma_2), x:\tau} by (4).

   \item[Case $\oplus_4$ ] As before, we can derive $\cwf{\Gamma_1}$ and \cwf{\Gamma_2}
     and thus, by induction hypothesis, \cwf{\Gamma_1 \oplus \Gamma_2}. As in the case
     $\oplus_3$, we are left to prove \ctwf{\Gamma_1 \oplus \Gamma_2}{\rho}. This can
     easily be derived by induction. 
 \end{description}
\end{proof}

\begin{prop}
  \label{prop:oplusabsorb}
  If $\capreg{\_}\notin \Gamma_2$, and $\Gamma_1 \oplus \Gamma_2$ is defined, then $\Gamma_1
  \oplus \Gamma_2 = \Gamma_1$.
\begin{proof}
  By induction of the construction of $\Gamma_1 \oplus \Gamma_2$.
  \begin{description}
    \item[Case $\oplus_1$] There is nothing to prove.

    \item[Case $\oplus_2$] We have $\capreg{\_}\notin \Gamma_2, \chi$, which gives us
      $\capreg{\_}\notin \Gamma_2$ and thus $\Gamma_1\oplus \Gamma_2 = \Gamma_1$ by induction
      hypothesis, and finally $(\Gamma_1,\chi) \oplus (\Gamma_2,\chi) = (\Gamma_1 \oplus \Gamma_2), \chi =
      \Gamma_1, \chi$.

    \item[Cases $\oplus_3$ and $\oplus_4$] Similar.
    
    \item[Case $\oplus_5$ ] Similar as well. Because of symmetry, we also
      should consider the case where $\Gamma_1$ and $\Gamma_2$ are inversed, but this
      leads to the contradiction $\capreg{\_}\notin \Gamma_2,\capreg{p}$. 
  \end{description}
\end{proof}
\end{prop}

\begin{prop}
  \label{prop:defplusplus}
  If $\Gamma_1 = \Gamma_a \oplus \Gamma_b$ and $\Gamma = \Gamma_1 \oplus \Gamma_2$ are defined, then 
  $\Gamma_a \oplus \Gamma_2$ is defined as well.
\end{prop}
%\begin{proof}
%  %TODO
%\end{proof}

\subsection{Typing}
\label{sec:typing}

We assume given a function $TypeOf : c \mapsto \sigma$ with the property that
$TypeOf(c) = \sigma$ implies $fv(\sigma) = \emptyset$. We have at least $TypeOf(\void) =
\unit$ and the following typings for functions manipulating references:

\begin{eqnarray*}
  TypeOf(\reft) & = & \forall\alpha\rho\epsilon.\regty{\rho}{\alpha\effarr{\epsilon\cup \rho} \refreg{\alpha}{\rho}} \\
  TypeOf( := ) & = & \forall\alpha\rho\epsilon\epsilon'. \refreg{\alpha}{\rho} \effarr{\epsilon} \alpha
  \effarr{\epsilon'\cup \rho} \unit \\
  TypeOf( !) & = & \forall\alpha\rho\epsilon. \refreg{\alpha}{\rho} \effarr{\epsilon\cup \rho} \alpha \\
\end{eqnarray*}

The typing rules are summarised in figure \ref{fig:bastyprules}.

\begin{figure}[htpb]
  \begin{center}
\begin{mathpar}
  {\inferrule* [left=\varrule] 
  {\Gamma(x) = \forall\bar{\chi}.\tau \quad \ctwf{\Gamma}{\bar{\xi}}\quad \capreg{\_}\notin \Gamma}
  {\typerel{\Gamma}{\tyapp{x}{\bar{\xi}}}{\tsub{\tau}{\bar{\chi}\mapsto\bar{\xi}}}{\emptyset}} }\and
  {\inferrule* [left=\conrule] 
    {TypeOf(c) = \tau \\ \cwf{\Gamma}\\ \capreg{\_}\notin \Gamma}
    {\typerel{\Gamma}{c}{\tau}{\emptyset}}}\and
  {\inferrule* [left=\absrule] 
    {\typerel{\Gamma,x:\tau'}{e}{\tau}{\varepsilon}\\ \capreg{\_}\notin\Gamma} 
    {\typerel{\Gamma}{\lambda x:\tau'.e}{\tau'\effarr{\varepsilon} \tau}{\emptyset}}}\and
  {\inferrule* [left=\apprule] 
    {\typerel{\Gamma_1}{e_1}{\tau'\effarr{\varepsilon} \tau}{\varepsilon_1}\\ \typerel{\Gamma_2}{e_2}{\tau'}{\varepsilon_2}} 
    {\typerel{\Gamma_1\oplus \Gamma_2}{e_1\ e_2}{\tau}{\varepsilon\cup \varepsilon_1 \cup \varepsilon_2}}}\and
  {\inferrule* [left=\letregrule] 
    {\typerel{\Gamma,\rho,\capreg{\rho}}{e}{\tau}{\varepsilon}} 
    {\typerel{\Gamma}{\letreg{\rho}{e}}{\tau}{\varepsilon \setminus \rho}}}\and
  {\inferrule* [left=\absregrule] 
    {\typerel{\Gamma,\capreg{\rho}}{e}{\tau}{\varepsilon}\\ \capreg{\_}\notin\Gamma} 
    {\typerel{\Gamma}{\regpr{\rho}{e}}{\regty{\rho,\varepsilon}{\tau}}{\emptyset}}}\and
  {\inferrule* [left=\appregrule] 
    {\typerel{\Gamma,\rho}{e}{\regty{\rho,\varepsilon'}{\tau}}{\varepsilon}}
    {\typerel{\Gamma,\rho,\capreg{\rho}}{\rapp{e}{\rho}}{\tau}{\varepsilon\cup \varepsilon'}}} \and
  {\inferrule* [left=\letrule]
    {\typerel{\Gamma_1,\bar{\alpha}\bar{\rho}\bar{\epsilon}}{v}{\tau'}{\emptyset} \\
     \typerel{\Gamma_2,x : \forall\bar{\alpha}\bar{\rho}\bar{\epsilon}.\tau'}{e}{\tau}{\varepsilon}}
    {\typerel{\Gamma_1 \oplus \Gamma_2}{\mllet{\bar{\alpha}\bar{\rho}\bar{\epsilon}\ x}{v}{e}}{\tau}{\varepsilon}}
  }
\end{mathpar}
  \end{center}
  \caption{The Core Typing Rules}
  \label{fig:bastyprules}
\end{figure}

\begin{figure}[htpb]
  \begin{center}
    \begin{mathpar}
      {\inferrule* [left=\regionrule] 
        {\typerel{\Gamma}{e}{\tau}{\varepsilon}} 
        {\typerel{\Gamma}{\region{\rho}{e}}{\tau}{\varepsilon \setminus \rho}}}\and
      {\inferrule* [left=\locrule] 
        {\Gamma(y) = \refreg{\tau}{\rho}\and \cwf{\Gamma} \and \capreg{\_}\notin\Gamma}
        { \typerel{\Gamma}{y_\rho}{\refreg{\tau}{\rho}}{\emptyset}}} \and
      {\inferrule* [left=\subeffrule] 
        {\typerel{\Gamma}{e}{\tau}{\varepsilon}\\ \varepsilon \subeff \varepsilon' \and \ctwf{\Gamma}{\varepsilon'}} 
        {\typerel{\Gamma}{(e : \varepsilon')}{\tau}{\varepsilon'}}}
    \end{mathpar}
  \end{center}
  \caption{Bookkeeping typing rules}
  \label{fig:advtyprules}
\end{figure}


\begin{prop}
  \label{prop:capvalues}
  $\typerel{\Gamma}{v}{\tau}{\varepsilon}$ for any value $v$ implies {\normalfont
  $\capreg{\_}\notin\Gamma$} and $\typerel{\Gamma}{v}{\tau}{\emptyset}$.
\end{prop}
\begin{proof}
  By looking at the premises for the rules for values: \varrule, \conrule,
  \absrule{} and \absregrule.
\end{proof}

\begin{prop}
  \typerel{\Gamma}{e}{\tau}{\varepsilon} implies \cwf{\Gamma} and \ctwf{\Gamma}{\tau,\varepsilon}.
\end{prop}
\begin{proof}
  We proceed by induction on the typing derivation. 
  \begin{description}

    \item[Cases \varrule, \conrule{} and \locrule] $\cwf{\Gamma}$ is a premise of
      these rules. This implies the other claims as well.

    \item[Cases \absrule{} and \absregrule] By applying the induction
      hypothesis to the pre\-mise of any of the rules, we have \cwf{\Gamma, \nu},
      where $\nu$ is the object (program variable or region capability)
      introduced by the corresponding rule. By proposition
      \ref{prop:cwfprefix}, we can conclude.

    \item[Case \apprule] By applying proposition \ref{prop:cwfoplus} and the
      induction hypothesis.

    \item[Case \subeffrule] By induction hypothesis.

    \item[Cases \letregrule{} and \regionrule] By proposition
      \ref{prop:cwfprefix} and the induction hypothesis.

    \item[Case \appregrule] by applying the rule (3). The needed hypotheses
      are obtained by induction hypothesis (\cwf{\Gamma,p} implies \cwf{\Gamma} and
      \ctwf{\Gamma}{p}).

    \item[Case \letrule] By combining the rules (2),(8) and proposition
      \ref{prop:cwfoplus}.

  \end{description}
\end{proof}

\begin{lem}[Variable substitution lemma]
  \newcommand{\thesub}[1]{\tsub{#1}{\chi' \mapsto \xi}} 
  \label{lem:typesubst}
  \typerel{\Gamma,\chi',\Delta}{e}{\tau}{\varepsilon} and \ctwf{\Gamma}{\xi} imply 
  \typerel{\Gamma,\thesub{\Delta}}{\thesub{e}}{\thesub{\tau}}{\thesub{\varepsilon}}.
\end{lem}
\begin{proof}
  Let $\theta$ be the substitution $\left[ \chi'\mapsto \xi \right]$.
  \newcommand{\thesub}[1]{\ensuremath{#1 \theta}}
  We proceed by induction over the typing derivation of
  \typerel{\Gamma,\chi',\Delta}{e}{\tau}{\varepsilon}.  We use the facts $\chi'\notin fv(\Gamma)$ (by
  proposition \ref{prop:cwftysub}) and $\chi'\notin\xi$ (by proposition
  \ref{prop:cwffvty}).
  \begin{description}

    \item[Case \varrule] The conclusion of the rule is
      \typerel{\Gamma,\chi',\Delta}{x}{\tau}{\emptyset}. Its premises are $(\Gamma,\chi',\Delta)(x) = \tau$
      (1) and \cwf{\Gamma,\chi',\Delta}. By proposition \ref{prop:cwftysub}, we obtain
      $\cwf{\Gamma,\thesub{\Delta}}$. If $x\in \Gamma$, then $(\Gamma,\chi',\Delta)(x) = \Gamma(x) =
      (\Gamma,\thesub{\Delta})(x) = \tau = \thesub{\tau}$, as $\chi'\notin \Gamma$ and by proposition
      \ref{prop:cwffvty}. If $x\in \Delta$, then $\Delta(x) = \tau$ and $(\thesub{\Delta})(x) =
      \thesub{\tau}$ and we can conclude.

    \item[Case \conrule] obvious, as $\tau$ does not contain free variables. 

    \item[Case \absrule] The conclusion of the rule is \typerel{\Gamma,\chi',\Delta}{\lambda
      x:\tau'.e}{\tau'\effarr{\varepsilon} \tau}{\emptyset}. Its premises are
      \typerel{\Gamma,\chi,\Delta,x:\tau'}{e}{\tau}{\varepsilon} (1) and $\capreg{\_} \notin \Gamma,\chi',\Delta$.  We
      can immediately conclude $\capreg{\_} \notin \Gamma,\thesub{\Delta}$, and by
      induction hypothesis, we have
      \typerel{\Gamma,\thesub{\Delta},x:\thesub{\tau'}}{\thesub{e}}{\thesub{\tau}}{\varepsilon}.  We
      get, by application of \absrule{}, \typerel{\Gamma,\thesub{\Delta}}{\thesub{(\lambda
      x:\tau'.e)}}{\thesub{(\tau'\effarr{\varepsilon} \tau)}}{\emptyset}, which is the statement
      we wanted.

    \item[Case \recrule] similar. 
      
      
    \item[Cases \absregrule] Similar. 


    \item[Case \subeffrule] by induction hypothesis. 

    \item[Cases \appregrule, \letregrule, \regionrule{} and \locrule] By
      reconstructing the subs\-ti\-tu\-ted de\-ri\-vation tree, as in the case
      for \absrule{}.  
  
    \item[Case \letrule] By induction hypothesis. One can assume $\chi'$ and
      $fv(\xi)$ different from the $\bar{\alpha},\bar{\rho},\bar{\epsilon}$ of the rule.

  \end{description} 
\end{proof}

\begin{prop}[Irrelevant hypotheses]
  \label{prop:irrelhyp}
  If \typerel{\Gamma,x : \tau',\Delta}{e}{\tau}{\varepsilon} and $x\notin fv(e)$, then
  \typerel{\Gamma,\Delta}{e}{\tau}{\varepsilon}.
\end{prop}
\begin{proof}
  By induction on the typing derivation. This proof poses no difficulties.
\end{proof}

\begin{prop}[Weak Weakening]
  \label{prop:weakweak}
  If $\Gamma_1 \oplus \Gamma_2$ is defined, then \typerel{\Gamma_1}{e}{\tau}{\varepsilon} implies
  \typerel{\Gamma_1 \oplus \Gamma_2}{e}{\tau}{\varepsilon}.
\end{prop}

\begin{prop}[Strong Weakening]
  \label{prop:strongweak}
  If $\typerel{\Gamma}{e}{\tau}{\varepsilon}$ and $\ctwf{\Gamma}{\Delta}$ and $\capreg{\_}\notin \Delta$, we
  have $\capreg{\rho}\notin \Delta$, then \typerel{\Gamma,\Delta}{e}{\tau}{\varepsilon}.
\end{prop}

\begin{prop}[General Exchange]
  \label{prop:genexchange}
  The following statements are true:
  \begin{itemize}
    \item $\typerel{\Gamma, x : \tau, y : \tau', \Delta}{e}{\tau''}{\varepsilon}\Leftrightarrow \typerel{\Gamma, y :
      \tau', x : \tau, \Delta}{e}{\tau''}{\varepsilon}$.
   \item  $\typerel{\Gamma, x : \tau, \chi , \Delta}{e}{\tau''}{\varepsilon}\Leftrightarrow\typerel{\Gamma,\chi, x
     : \tau, \Delta}{e}{\tau''}{\varepsilon}$, if $\chi\notin fv(\tau)$.
   \item  $\typerel{\Gamma, x : \tau, \capreg{\rho} ,
     \Delta}{e}{\tau''}{\varepsilon}\Leftrightarrow\typerel{\Gamma,\capreg{\rho}, x : \tau, \Delta}{e}{\tau''}{\varepsilon}$
 \end{itemize}
\end{prop}
\begin{proof}
  straightforward induction over the typing derivation, observing that the
  respective environments are always well-formed.
\end{proof}

\begin{lem}[Substitution Lemma]
  \label{lem:substlem}
  If $\Gamma = \Gamma_1 \oplus \Gamma_2$ is defined and
  $\capreg{\_}\notin \Gamma_2$, then \typerel{\Gamma_1,x:\tau'}{e}{\tau}{\varepsilon} and
  \typerel{\Gamma_2}{v}{\tau'}{\varepsilon'} imply \typerel{\Gamma_1}{\esub{e}{x\mapsto
  v}}{\tau}{\varepsilon}.
\end{lem}
\begin{proof}
  Let us call $\sigma$ the substitution $\left[ x \mapsto v \right]$.  We proceed
  by induction on the derivation tree for \typerel{\Gamma_1,x:\tau'}{e}{\tau}{\varepsilon}.
  \begin{description}

    \item[Case \varrule] This means that $e$ is of the form $y$. If $x = y$,
      then we have $\tau' = \Gamma_1(x) = \Gamma_1(y) = \tau$. Using $x\sigma = v$, we obtain
      \typerel{\Gamma_1}{e\sigma}{\tau}{\emptyset} by the second premise and proposition
      \ref{prop:capvalues}. If $x \neq y$, we
      know that $e\sigma = y\sigma = y$, and we can conclude by proposition
      \ref{prop:irrelhyp}. 

    \item[Case \conrule] As in the case of \varrule, with $x \neq y$.

    \item[Case \absrule] The conclusion of the rule is \typerel{\Gamma_1,x:\tau'}{\lambda
      y:\tau''.e}{\tau''\effarr{\varepsilon} \tau}{\emptyset}. We can assume that $y \neq x$. The
      premises of the rule are \typerel{\Gamma_1,x:\tau',y:\tau''}{e}{\tau}{\varepsilon} and
      $\capreg{\_}\notin \Gamma_1,x:\tau'$ (a).  By proposition
      \ref{prop:genexchange}, we obtain \typerel{\Gamma_1,y:\tau'',x:\tau'}{e}{\tau}{\varepsilon}. As
      $\Gamma_1 \oplus \Gamma_2$ is defined, $\Gamma_1, y : \tau'') \oplus (\Gamma_2, y : \tau'')$ is
      defined as well and is equal to $(\Gamma_1 \oplus \Gamma_2), y : \tau''$. We also
      have $\typerel{\Gamma_2, y : \tau}{v}{\tau'}{\varepsilon'}$ by proposition
      \ref{prop:strongweak}, and by induction hypothesis we have
      \typerel{\Gamma,y:\tau''}{e\sigma}{\tau}{\varepsilon'}. Now we know that $\capreg{\_}\notin \Gamma$
      (by (a) and proposition \ref{prop:capvalues}) and we can conclude with
      the rule \absrule, and by $\lambda y:\tau''.e\sigma = (\lambda y:\tau''.e)\sigma$.

    \item[Case \apprule] The conclusion of the rule is $\typerel{\Gamma_1, x :
      \tau'}{e_1\ e_2}{\tau}{\varepsilon \cup \varepsilon_1 \cup \varepsilon_2}$. But we also have $\Gamma_1, x : \tau' =
      \Gamma_a \oplus \Gamma_b$.  It can be seen that we have $\Gamma_a = \Gamma_a', x:\tau'$ and
      $\Gamma_b = \Gamma_b', x : \tau'$. The premises of the rule are $\typerel{\Gamma_a', x :
      \tau'}{e_1}{\tau'' \effarr{\varepsilon} \tau}{\varepsilon_1}$ and $\typerel{\Gamma_b',
      x:\tau'}{e_2}{\tau''}{\varepsilon_2}$. By induction hypothesis and proposition
      \ref{prop:defplusplus}, we obtain $\typerel{\Gamma_a'\oplus \Gamma_2}{e_1
      \sigma}{\tau''\effarr{\varepsilon} \tau}{\varepsilon_1}$ and $\typerel{\Gamma_b'}{e_2 \sigma}{\tau''}{\varepsilon_2}$, and
      by the rule \apprule{} we have $\typerel{(\Gamma_a'\oplus \Gamma_2) \oplus (\Gamma_b'
      \oplus \Gamma_2')}{(e_1\ e_2) \sigma}{\tau}{\varepsilon\cup\varepsilon_1\cup\varepsilon_2}$. Observing that
      $(\Gamma_a'\oplus \Gamma_2) \oplus (\Gamma_b' \oplus \Gamma_2) = \Gamma_1 \oplus \Gamma_2 = \Gamma$, we can
      conclude.

    \item[Cases \letregrule, \regionrule{} and \locrule{}] By induction
      hypothesis and proposition \ref{prop:genexchange}. 

    \item[Case \absregrule] The conclusion of the rule is \typerel{\Gamma, x :
      \tau'}{\regpr{p}{e}}{\regty{p,\varepsilon}{\tau}}{\emptyset}. Its premises are
      \typerel{\Gamma, x : \tau', \capreg{p}}{e}{\tau}{\varepsilon} and $\capreg{\_}\notin \Gamma$. By
      general exchange, we can derive \typerel{\Gamma, \capreg{p}, x : \tau'}{e}{\tau}{\varepsilon}
      and thus, by induction hypothesis, \typerel{\Gamma, \capreg{p}}{e\sigma}{\tau}{\varepsilon}. By
      \absregrule{}, we obtain
      \typerel{\Gamma}{\regpr{p}{e}}{\regty{p,\varepsilon}{\tau}}{\emptyset}, as desired.  
    
    \item[Case \appregrule] Impossible 
  
    \item[Case \letrule] Simply by induction hypothesis. One can assume
      that $\tau'$ does not contain any of the bound variables $\chi$.
  \end{description}
\end{proof}

\subsection{Subject Reduction}
\label{sec:subjectreduction}

\begin{definition}
  For an environment $\Gamma$, define 
  \begin{eqnarray*}
    dom_\rho(\Gamma) &=& \left\{ \rho \mid \rho\in \Gamma \right\}, \\
    \Gamma(\rho) &=& \left\{ y \mapsto \tau \mid y : \refreg{\tau}{\rho} \in \Gamma\right\}.
  \end{eqnarray*}
  Now define $\storetype{\Gamma}{s}$ iff
  \begin{itemize}
    \item $dom_\rho(\Gamma) = dom(s)$
    \item $\forall \rho\in dom(s), dom(\Gamma(\rho)) = dom(s(\rho))$
    \item $\forall \rho\in dom(s), \forall y\in dom(s(\rho)),
      \typerel{\Gamma}{s(\rho)(y)}{\Gamma(\rho)(y)}{\emptyset}$.
  \end{itemize}
\end{definition}


\begin{definition}
  \label{def:invariants}
  For an environment $\Gamma$, a store $s$, an expression $e$, a type $\tau$ and an
  effect $\varepsilon$, we define the following four properties:
  \begin{description}
    \item[(a)] $\typerel{\Gamma}{e}{\tau}{\varepsilon}$
    \item[(b)] $\storetype{\Gamma}{s}$
    \item[(c)] $\capreg{\rho} \in \Gamma$ implies $s(\rho) = \emptyset$
    \item[(d)] $singleton(s)$, that is $\forall\rho\in dom(s)$, either $s(\rho) =
      \emptyset$ or $s(\rho) = \{ y \mapsto v\}$ for some $y,v$.
  \end{description}
\end{definition}

\begin{thm}[Subject Reduction]
  \label{thm:subred}
  Suppose the properties of \ref{def:invariants} are true for $\Gamma,s,e,\tau,\varepsilon$ and
  suppose $s,erase(e) \basered s',erase(e')$ for some $s',e'$. Then there is
  an environment $\Gamma'$ such that these properties are true as well for
  $\Gamma',s',e',\tau,\varepsilon$. $\Gamma'$ relates to $\Gamma$ in one of the following ways:
  \begin{itemize}
    \item $\Gamma' = \Gamma, \rho,\capreg{\rho}$ with $\rho\notin \Gamma$
    \item $\Gamma' = (\Gamma - \capreg{\rho}), y : \refreg{\tau}{\rho}$
    \item $\Gamma' = \Gamma$ 
  \end{itemize}
\end{thm}
\begin{proof}
  By induction on the reduction $s, e \basered s', e'$. In these subcases we
  usually disregard the possibility that the last rule applied to establish
  the typing of $e$ has been the $\subeffrule$ rule. This does not alter the
  proofs.   \begin{description}
    \item[Case $\beta$] In this case, $e = (\lambda x : \tau'. e_1)\ v$ and we have the
      reduction $s,(\lambda x : \tau'. e_1)\ v \basered s, \esub{e}{x\mapsto v}$.
      Every derivation tree for $\typerel{\Gamma}{(\lambda x : \tau'. e_1)\ v}{\tau}{\varepsilon}$ has
      the following shape:
      \begin{equation*}
      \inferrule* [left=\apprule]
        {\inferrule* [left=\subeffrule] 
          {\inferrule* [left = \absrule]
            {\typerel{\Gamma_1, x : \tau'}{e_1}{\tau}{\varepsilon} \\ \capreg{\_}\notin \Gamma_1}
            {\typerel{\Gamma_1}{\lambda x : \tau'.e_1}{\tau'\effarr{\varepsilon} \tau}{\emptyset}
            } }
        {\typerel{\Gamma_1}{\lambda x : \tau'.e_1}{\tau'\effarr{\varepsilon} \tau}{\varepsilon_1}} \\
        {\typerel{\Gamma_2}{v}{\tau'}{\varepsilon_2} }}
        {\typerel{\Gamma_1}{(\lambda x : \tau'. e_1)\ v}{\tau}{\varepsilon \cup \varepsilon_1 \cup \varepsilon_2}}
      \end{equation*}

      As a remark, we are allowed to write $\Gamma_1$ instead of $\Gamma_1\oplus \Gamma_2$ in
      the last line thanks to propositions \ref{prop:capvalues} and
      \ref{prop:oplusabsorb}. We still can use the fact that $\Gamma_1 \oplus \Gamma_2$
      is defined. This will also be used in many of the other cases.

      Now the substitution lemma \ref{lem:substlem} uses the established facts
      to exactly prove (a) : \typerel{\Gamma}{\esub{e}{v \mapsto v}}{\tau}{\varepsilon}. Using
      the \subeffrule{} rule, we can increase the effect as desired to $\varepsilon \cup
      \varepsilon_1 \cup \varepsilon_2$. We now can take $s' = s$ and the properties (b), (c) and
      (d) are true by hypothesis. Also note that if the last rule applied was
      an additional \subeffrule{} rule, we could have increased the effect
      accordingly in the resulting typing derivation, by the bias of the last
      typing rule.

    \item[Case $\chi$] In this case, $e = \tyapp{(\lambda\chi.e)}{\xi}$, and we have the
      following reduction: $s, \tyapp{(\lambda\chi.e)}{\xi} \basered s,
      \tsub{e}{\chi\mapsto \xi}$.  Every derivation tree for
      $\typerel{\Gamma}{\tyapp{(\lambda\chi.e)}{\xi}}{\tau}{\varepsilon}$ has the following shape:
      \begin{equation*}
        \inferrule* [left=\apptyrule]
          {\inferrule* [left=\abstyrule]
            {\typerel{\Gamma,\chi}{v}{\tau}{\emptyset} \\ \capreg{\_}\notin \Gamma}
            {\typerel{\Gamma}{\lambda\chi.v}{\forall\chi.\tau}{\emptyset} \\ \ctwf{\Gamma}{\xi} } }
            {\typerel{\Gamma}{\tyapp{(\lambda\chi.e)}{\xi}}{\tsub{\tau}{\chi\mapsto \xi}}{\emptyset}}
      \end{equation*}
      Now the type substitution lemma \ref{lem:typesubst} gives us
      $\typerel{\tsub{e}{\chi \mapsto \xi}}{\tsub{\tau}{\chi\mapsto \xi}}{\emptyset}$, as
      desired. Again we can take $s' = s$ $\Sigma' = \Sigma$ and the properties (b), (c)
      and (d) are true by hypothesis.

\item[Case $p$] In this case, $e = \rapp{\regpr{p}{e}}{p}$, and we have the
  following reduction: $s, \rapp{\regpr{p}{e}}{p} \basered s,e$. The
  derivation tree for $\typerel{\Gamma}{\rapp{\regpr{p}{e}}{p}}{\tau}{\varepsilon}$ has the
  following shape:
  \begin{equation*}
    \inferrule* [left=\appregrule]
      { \inferrule* [left=\absregrule]
        {\typerel{\Gamma,p,\capreg{p}}{e}{\tau}{\varepsilon} \\ \capreg{\_}\notin \Gamma,\Sigma}
        {\typerel{\Gamma,p}{\regpr{p}{e}}{\regty{p,\varepsilon}{\tau}}{\emptyset} \\ \capreg{p}\notin \Gamma}}
        { \typerel{\Gamma,p,\capreg{p}}{\rapp{(\regpr{p}{e})}{p}}{\tau}{\varepsilon}}
  \end{equation*}
    The conclusion is now obvious.
    \item[Case $letreg$]
      In this case, $e = \letreg{\rho}{e'}$ and we have the following reduction:
      $s,\letreg{\rho}{e'} \basered \statesub{s}{\rho_1\mapsto \left\{ \right\}},
      \region{\rho_1}{\regsub{e'}{\rho\mapsto \rho_1}}$, where $\rho_1\notin s$. The
      derivation tree looks as follows:
    \begin{equation*}
      \inferrule* [left=\letregrule]
        {\typerel{\Gamma,\rho,\capreg{\rho}}{e'}{\tau}{\varepsilon} \and \capreg{\rho}\notin \Gamma}
        {\typerel{\Gamma}{\letreg{\rho}{e'}}{\tau}{\varepsilon\setminus \rho }}
    \end{equation*}

    Using the \regionrule{} rule, and defining $\Gamma' = \Gamma, \rho_1, \capreg{\rho_1}$, we
    obtain (a).  Property (d) is still true for $s'$, as we only added an
    empty region.  Property (c) is correct for $\Gamma'$ and $s'$, as we added one
    capacity to $\Gamma$, whose region is empty. Finally, $(b)$ is correct, because
    we added the same to the store as to the environment.

  \item[Case $region$] In this case, $e = \region{\rho_1}{v}$ and we have the
    following reduction: $s, \region{\rho_1}{v} \basered s,v$. The derivation
    tree for $\typerel{\Gamma}{\region{\rho_1,\rho}}{\tau}{\varepsilon}$ looks as follows:
    \begin{equation*}
      { \inferrule* [right=\regionrule]
        { \typerel{\Gamma}{v}{\tau}{\varepsilon} }
        { \typerel{\Gamma}{\region{\rho_1}{v}}{\tau}{\varepsilon\setminus \rho_1} }
        }
    \end{equation*}
    Taking $\Gamma' = \Gamma$, (a) is trivial as the type $\tau$ is preserved and $v$ is a
    value. As $s' = s$, the other conditions are true by hypothesis.
  \item[Case $ref$] In this case, $e = \reft \left[ \xi_1 \dots \xi_n \right]
    \capreg{\rho_1}\ v$, and we have the following reduction starting from $s,e$:
    $s, (\reft \left[ \xi_1 \dots \xi_n \right] \ \capreg{\rho_1})\ v \basered
    s\left[ \rho_1 \mapsto s(\rho_1)\left[ y\mapsto v \right]\right], y_{\rho_1}$ where
    $\rho_1\in s, y\notin s(\rho_1)$. We have the following derivation tree:
    \begin{equation*}
    { \inferrule* [right=\apprule]
      { \inferrule* [right=\appregrule]
        {\typerel{\Gamma_1}
                   {\reft \left[ \xi_1 \dots \xi_n \right] }
                   {\regty{\rho,\varepsilon'}{(\tau\effarr{\varepsilon}\refreg{\tau}{\rho})}}
                   {\varepsilon_1} }
        {\typerel{\Gamma_1, \capreg{\rho_1}}
                 {(\reft \left[ \xi_1 \dots \xi_n \right] )@\rho_1}
                 {\tau\effarr{\varepsilon}\refreg{\tau}{\rho}}{\varepsilon'\cup \varepsilon_1}} \\
                 \typerel{\Gamma_2}{v}{\tau}{\varepsilon_2} 
      }
      {\typerel{\Gamma_1, \capreg{\rho_1}}{\reft \left[ \xi_1 \dots \xi_n \right] \rho_1\ v}{\refreg{\tau}{\rho}}{\varepsilon\cup \varepsilon_1 \cup \varepsilon_2 \cup \varepsilon'}}
    }
    \end{equation*}
      with $\rho_1\in \Gamma_1$. Now, define $\Gamma' = (\Gamma - \capreg{\rho_1}), y :
      \refreg{\tau}{\rho_1}$. By construction, we have (a) (by application of the
      \locrule{} rule). We also easily have (c): The store typing only changed
      for $\rho_1$, but we removed the capacity $\capreg{\rho_1}$ from $\Gamma$. We also
      have (d): As $\capreg{\rho_1}\in \Gamma$, we have by hypothesis that $s(\rho_1) =
      \emptyset$. Now, $s'$ differs from $s$ only in a reference allocated in
      $\rho_1$, so $singleton(s')$ is still true. For (b), we know that $\rho_1\in
      dom(s)$, so $dom(s) = dom(s')$, and thus $dom_\rho(\Gamma') = dom(s')$.  We now
      have $y\in dom(\Gamma'(\rho_1))$, but we also have $y\in dom(s'(\rho_1))$. As
      $\typerel{\Gamma_2}{v}{\tau}{\varepsilon_2}$, by proposition \ref{prop:weakweak}, we
      have $\typerel{\Gamma'}{v}{\tau}{\varepsilon_2}$. By construction, we have $\Gamma'(\rho_1)(y) =
      \tau$.  This proves $\typerel{\Gamma'}{s'(\rho_1)(y)}{\Gamma'(\rho_1)(y)}{\emptyset}$.

    \item[Case $:=$] In this case, $e = (:= \left[ \xi_1 \dots \xi_n \right]
      y_{\rho_1}\  v)$ and we have the following reduction $s, := \left[ \xi_1
      \dots \xi_n \right] \ y_{\rho_1}\ v \basered s\left[ \rho_1 \mapsto
      s(\rho_1)\left[ y \mapsto v \right] \right], \void$, with $\rho_1\in s, y\in
      s(\rho_1)$. The derivation tree for $\typerel{\Gamma}{e}{\tau}{\varepsilon}$ looks as
      follows:
      \begin{equation*}
        { \inferrule* [right=\apprule]
          { 
            \inferrule* [right=\apprule]
              {\typerel{\Gamma_1}
                       { := \left[ \xi_1 \dots \xi_n \right] }
                       { \refreg{\tau}{\rho} \effarr{\varepsilon} \tau \effarr{\varepsilon'} \unit }
                       {\varepsilon_1}
                \\
              \typerel{\Gamma_1}{y_{\rho_1}}{\refreg{\tau}{\rho}}{\varepsilon_2}\  (1) }
              {\typerel{\Gamma_1}
                    {:= \left[ \xi_1 \dots \xi_n \right] y_{\rho_1}}
                    { \tau \tyarr \unit}
                    {\varepsilon\cup \varepsilon_1 \cup \varepsilon_2}
               } \\ 
             \typerel{\Gamma_2}{v}{\tau}{\varepsilon_3} \   (2)
          }
          { \typerel{\Gamma}
                    {:= \left[ \xi_1 \dots \xi_n \right] y_{\rho_1}\ v)}{\unit}
                    {\varepsilon \cup \varepsilon_1 \cup \varepsilon_2 \cup \varepsilon_3 \cup \varepsilon'}
          }
        }
      \end{equation*}
      Set $\Gamma' = \Gamma$. As $e'$ is \void, (a) is trivial to prove by the
      \conrule{} and \subeffrule{} rules. For (b), it suffices to show that
      $\typerel{\Gamma}{s'(\rho_1)(y)}{\Gamma(\rho_1)(y)}{\emptyset}$. Using (1) and (2), this
      is obvious. (c) and (d) are true by hypothesis.

    \item[Case $!$]
      In this case, $e = ! \left[\xi_1 \dots \xi_n\right] y_{\rho_1}$, and we have
      the reduction $s, ! \left[\xi_1 \dots\xi_n\right] y_{\rho_1} \basered s,
      s(\rho_1)(y)$, where $\rho_1\in s, y\in s(\rho_1)$. The derivation tree for
      $\typerel{\Gamma}{e}{\tau}{\varepsilon}$ looks as follows:
      \begin{equation*}
        { \inferrule* [right=\apprule]
          { \typerel{\Gamma}{!\left[\xi_1 \dots\xi_n\right]}
            {\refreg{\tau}{\rho} \effarr{\varepsilon} \tau}{\varepsilon_1} \\
            \typerel{\Gamma} { y_{\rho_1}} {\refreg{\tau}{\rho} }{\varepsilon_2}\ (1)}
          { \typerel{\Gamma} {! \left[\xi_1 \dots\xi_n\right] y_{\rho_1}}{\tau}{\varepsilon\cup
          \varepsilon_1\cup \varepsilon_2} }
        }
      \end{equation*}
      Again, we set $\Gamma' = \Gamma$. As $s' = s$, we have the properties (b), (c),
      and (d) for free by hypothesis. Using (1), we easily prove (a).
  \end{description}
\end{proof}

\begin{thm}[One-Step Subject Reduction]
  If the properties of definition \ref{def:invariants} are true for $\Gamma,e,\tau,s$
  and $\varepsilon$, then for any $s',e'$ such that $s,e\onestep s',e'$ there is a $\Gamma'$  
  such that these properties are true for $\Gamma',e',\tau,s'$ and $\varepsilon$. $\Gamma'$ relates
  to $\Gamma$ as in theorem \ref{thm:subred}
  \label{thm:onestep-subred}
\end{thm}
\begin{proof}
  By induction on the form of the reduction context.
  \begin{description}
    \item[Case \bracket{~} ] This corresponds to the theorem
      \ref{thm:subred}.

    \item[Case $E\ \bracket{\xi}$ ] There is nothing to prove: The rule
      \appgenrule{} applies as well after the reduction.

    \item[Case $E\ e$] We know that $e = e_1\ e_2$ and $s,e_1 \onestep
      s',e_1'$. The typing derivation looks as follows :
      \begin{equation*}
        { \inferrule* [left=\apprule]
          { \typerel{\Gamma_1}{e_1}{\tau' \effarr{\varepsilon} \tau}{\varepsilon_1} \and
          \typerel{\Gamma_2}{e_2}{\tau'}{\varepsilon_2}}
          { \typerel{\Gamma_1 \oplus \Gamma_2}{e_1 \oplus e_2}{\tau}{\varepsilon\cup\varepsilon_1\cup\varepsilon_2 } }
        }
      \end{equation*}
      We can apply the induction hypothesis to obtain $\typerel{\Gamma_1'}{e_1'}{\tau,
      \tyarr \tau}{\varepsilon_1}$.  To reconstruct a typing for $e_1'\ e_2$, we need to
      show that we can extend $\Gamma_2$ to $\Gamma_2'$ such that $\Gamma_1'\oplus \Gamma_2'$ is
      defined and $e_2$ is still typable. Let $\Gamma_2'$ be obtained in the same
      way from $\Gamma_2$ as $\Gamma_1'$ from $\Gamma$, but without adding (or removing) any
      capabilities. Using proposition \ref{prop:strongweak}, we see that $e_2$
      is still typable using $\Gamma_2'$, and it is easy to see that $\Gamma_1' \oplus
      \Gamma_2'$ is defined.

    \item[Case $v\ E$] Similar to the previous case.

    \item[Case $\rapp{E}{p}$]  Similar, the typing rule \appregrule{} does not
      pose any problems.
  \end{description}
\end{proof}

\begin{thm}[General Subject Reduction]
  If $\Gamma,s,e, \tau,\varepsilon$ fulfill the properties of definition \ref{def:invariants},
  and if $s,e \reduction s',e'$, there is $\Gamma'$  such that $\Gamma',s,e,\tau,\varepsilon$ do
  fulfill these properties as well.
  \label{thm:generalsubred}
\end{thm}
\begin{proof}
  Simple induction over the transitive closure of $\onestep$ (over the length
  of the reduction sequence). In the reflexive
  case (0 steps), we have $s = s', e = e'$ and the claim is trivial. In the
  case of a single $\onestep$ step, theorem \ref{thm:onestep-subred} says
  exactly what we want. In the transitive case, we just apply the induction
  hypothesis twice.
\end{proof}

In particular, the properties of definition \ref{def:invariants} are true for
$\Gamma = \emptyset$, and $s = \emptyset$, and any well-typed closed
($\Gamma = \emptyset$) program $e$ of type $\tau$ and effect $\varepsilon$.

\subsection{Other properties}

\begin{lem}[Values]
  The only Values of type $\refreg{\tau}{p}$ are variables and locations.
  \label{lem:valueprops}
\end{lem}
\begin{proof}
  Simply by looking at the typing rules.
\end{proof}


\section{The Weakest Precondition Calculus}

In this section we prove the correctness of the weakest precondition calculus,
including the instantiation restriction. Instead of the dealing directly with
region calculus, we take for granted that a region corresponds to a reference
and use globally created references instead. While the weakest precondition
calculus is insensible to aliasing questions, it is not presented as such
here; we give a direct interpretation of the the store types $ε$ (written
$\langle ε \rangle$ in the paper), by defining a logic called {\em effect
logic}, which in turn can be translated to standard higher-order logic. In
effect logic, all operations on stores (the combine operator $\oplus$ and the
restriction operator $|_ε$) are carried out directly and this needs the
aliasing restriction. Unfortunately, this makes the presented proof
unnecessarily dependent on the aliasing restriction.  Our proof thus only
applies in the context of alias-free programs. However, it would be easy to
adapt it to the more general context. There are some other superficial
differences with language in the paper. Here we separate more clearly between
the logic and programs.  Other small differences are that here the {\bf let}
and {\bf if}-typing rules do effect union, while in the paper the subeffecting
rule takes care of that.

\subsection{The Language}

\subsubsection*{Syntax} 

  \begin{eqnarray*}
    c &::=& n \sep () \sep \true \sep \false\\
    v &::=& c \sep x~\alist{κ} \sep \lambda x\colonτ.e \sep
    R \sep \rec~(y\colonτ')~(x\colonτ).~e\\
    e &::=& v \sep v~v\sep \letst~x~\alist{χ}=e~\inst~e \sep v := v\sep
    !\,v\\
    & & \ifst~v~\thenst~e~\elsest~e\\
    r &::=& R \sep ρ \\
    b &::=& r \sep \ev\\
    ε  &::=& b,\cdots, b\\
    χ &::=& ρ \sep \ev \\
    κ &::=& τ \sep r \\
    \iota &::& \inttype \sep \unit \sep \bool \\
    τ  &::=& \iota \sep τ  ->^ε  τ \sep \reftype_r \\
    S  &::=& \forall \alist{χ}.τ \\
    E &::=& [~]\sep \letst~x~\alist{χ}=E~\inst~e\\
    Γ &::=& \emptyset \sep Γ, χ \sep Γ, x : S
  \end{eqnarray*}

\subsubsection*{Semantics}
%TODO lookup and assignment to type
  \framebox{$ s,e--\ s',e'$}
  \begin{mathpar}
    {\inferrule* [left=Beta] {~} {s, (\lambda x\colonτ.e)~v --\ s, e[x|->v]} }
    \and
    {\inferrule* [left=Rec] {~} {s,
        (\rec~(y\colonτ ')~(x\colonτ ).~e)~v --\ s,
        e[x|->v,y|->\rec~(y\colonτ ')~(x\colonτ ).~e]} }
    \and
    {\inferrule* [left=Let] {~} {s, \letst~x~\alist{χ}=v~\inst~e --\ s,
        e[\Lambda \alist{χ}.x|->v]} }
    \and {\inferrule* [left=Lookup] {~} {s, !\,r --\ s, s(r)} }
    \and {\inferrule* [left=Assign] {~} {s, r:=n --\ s[r:=n], \void} }
    \and {\inferrule* [left=IfTrue] {~}
      {s,\ifst~\true~\thenst~e_1~\elsest~e_2 --\ s,e_1} }
    \and {\inferrule* [left=IfFalse] {~}
      {s,\ifst~\false~\thenst~e_1~\elsest~e_2 --\ s,e_2} }
  \end{mathpar}
  \framebox{$ s,e--> s',e'$}
  \begin{mathpar}
    {\inferrule* [left=Context] {s,e --\ s',e'} 
      {s, E[e] --> s',E[e']} }
  \end{mathpar}
  \framebox{$ s,e->> s',e'$}
  \begin{mathpar}
    {\inferrule* [left=Refl] {~} {s,e ->> s,e} }
    \and {\inferrule* [left=Step] {s,e->>s',e' \\ s',e' ->> s'',e''} 
      {s,e ->> s'',e''} }
  \end{mathpar}

\subsubsection*{Typing}
%TODO define compatibility for ref substitution 
\framebox{$Γ |-_vv:τ$}\\
  \begin{mathpar}
      {\inferrule* [left=Var] {Γ(x) = \forall\alist{χ}.τ \\
          \sigma = [\alist{χ}|->\alist{κ}]\\ \sigma\simτ }
        {Γ|-_v x~\alist{κ} :τ\sigma} } \and
    { \inferrule* [left=Const] {~} {Γ |-_v c : \mathit{TypeOf}(c) } } \and
    {\inferrule* [left=Ref] {~} {Γ|-R : \reftype_R}} \and
    { 
\inferrule* [left=Abs] 
      {Γ ,x\colon τ '|- e : τ ;ε  \\ \ceil{Γ
          ,x:τ '}|- p : ε -> \proptype;ε \\
      \ceil{Γ ,x\colon τ '},|-q: ε -> ε
      ->\ceil{τ }->\proptype;ε} 
      {Γ |-_v \lambda x\colon τ '.\{p\}e\{q\}:τ '->^ε τ } }
    \and
    { \inferrule* [left=Rec]
%{Γ|-_v \rec~(y:\colonτ'->^ε} }
      {Γ ,y\colon τ '->^ετ ,~x\colon τ '|- e : τ ;ε  \\
        \ceil{Γ ,x\colon τ '}|- p : ε
        ->\proptype;ε \\
      \ceil{Γ ,x\colon τ '}|-q:ε ->ε ->\ceil{τ }->\proptype;ε} 
      {Γ |-_v \rec~(y\colon τ '->^{ε}τ )~(x\colon τ) 
        ').\{p\}e\{q\}:τ '->^ε τ
      } }
    \end{mathpar}
    \framebox{$Γ |-e:τ;ε$}\\
    \begin{mathpar}
    {\inferrule* [left=Val] {Γ |-_v v : τ } {Γ |-v : τ ;\emptyset} }
    \and
    {\inferrule* [left=Lookup] {Γ|-_v v : \reftype_ρ } {Γ |-!\,v : \inttype;ρ} }
    \and
    {\inferrule* [left=Assign] 
      {Γ|-_v v_1 : \reftype_ρ \\Γ |-_v v_2 : \inttype} 
      {Γ |-v_1:=v_2 : \unit;ρ} 
    }
    \and
    {\inferrule* [left=App] {Γ |-_v v_1 : τ '->^ε τ  \\ Γ |-_v v_2 : τ '} 
      {Γ |-v_1~v_2 : τ ;ε } }
    \and
    {\inferrule* [left=Let] {Γ |- e_1 : τ ';ε _1 \\ Γ ,x:τ '|- e_2 : τ ;ε _2} 
      {Γ |-\letst~x = e_1~\inst~e_2 : τ ;ε _1ε _2} }
    \and
    {\inferrule* [left=Letv] 
      {Γ ,\alist{χ}|-_v e : τ ' \\ Γ ,x:\forall \alist{χ}.τ '|- e : τ ;ε } 
      {Γ |-\letst~x~\alist{χ} = v~\inst~e : τ ;ε } }
    \and
    {\inferrule* [left=If]
      {Γ |- v : \bool \\ Γ |- e_1 : τ \\ Γ |- e_2 :
        τ}
      {Γ |- \ifst~v~\thenst~e_1~\elsest~e_2 : τ}
    }
  \end{mathpar}

\subsection{SHOL}

\subsubsection*{Syntax}
  \begin{eqnarray*}
    f &::=& c \sep x \sep \lambda x\colon\theta  .f \sep
     \\
    & & f~f\sep f => f \sep f = f \sep (f,f) \sep\\
    & & \forall x\colon\theta  .f \sep \forall \alist{χ}.f \sep \pi_1 \sep
    \pi_2 \\
    \theta   &::=& \ev\sep \inttype \sep \unit \sep \proptype \sep \\
    & & \theta   -> \theta   \sep \theta *\theta \\
    \Delta  &::=& \emptyset  \sep \Delta , x \colon \theta   \sep \Delta ,\alist{χ}
  \end{eqnarray*}
\subsubsection*{Typing rules}
    \framebox{$\Delta|-f:\theta$}\\
  \begin{mathpar}
   { \inferrule* [left=Var] {\Delta (x) = \theta  } {\Delta |- x : \theta  } } 
   \and
   { \inferrule* [left=Const] { } {\Delta |- c : \mathit{TypeOf}(c)  } } 
   \and
   { \inferrule* [left=Abs] {\Delta ,x\colon\theta  |-f:\theta  '} {\Delta |- \lambda x\colon\theta  .f : \theta  ->\theta  '} }
   \and
   { \inferrule* [left=App] {\Delta |-f:\theta  '->\theta   \\ \Delta |-f':\theta  '} {\Delta |- f~f' : \theta  } }
   \and
   { \inferrule* [left=Eq] {\Delta |-f,f':\theta   } {\Delta |- f=f' : \proptype} }
   \and
   { \inferrule* [left=Impl] {\Delta |-f,f':\proptype } {\Delta |- f=> f' : \proptype} }
   \and
   { \inferrule* [left=Ty] {\Delta ,\alist{χ}|-f:\proptype } {\Delta |-
       \forall \alist{χ}.f : \proptype} }
   \and
   { \inferrule* [left=FA] {\Delta ,x\colon\theta  |-f:\proptype } {\Delta |- \forall x\colon\theta  .f : \proptype} }
   \and
   { \inferrule* [left=Pair] {\Delta |-f_i:\theta  _i\quad i =1,2} {\Delta |- (f_1,f_2): \theta  _1*\theta  _2} }
   \and
   { \inferrule* [left=Proj,right={~\textnormal i=1,2}] {\Delta |-f:\theta  _1*\theta  _2} {\Delta |- \pi _i~f: \theta  _i}}
  \end{mathpar}

\subsection{Effect logic}

\begin{figure*}[tpb]
  \begin{minipage}[t]{0.5\linewidth}
  \end{minipage}
  \begin{minipage}[t]{0.5\linewidth}
  \end{minipage}
  \caption{Syntax and typing of the new constructs}
  \label{fig:pholeff}
\end{figure*}
\subsubsection*{Syntax}
  \begin{eqnarray*}
    l & & \text{Labels} \\
    f &::=& ... \sep !\,r^l \sep x~\alist{ε }
    \sep 
    \lambda l\colonε.f \sep \\
    & & \forall l\colonε.f \sep f~s\\
    s &::=& \{r_1 = f_1 \cdots r_n = f_n, \ev_1 = l_1, \cdots, \ev_m
      = l_m \}\\
    \theta &::=& \iota \sep \proptype \sep \theta -> \theta \sep
    \theta*\theta \sep ε ->\theta \\
    \zeta &::=& \forall \alist{χ}.\theta \\
    \Delta  &::=& \emptyset \sep \Delta ,x\colon\zeta \sep \Delta ,\alist{χ}
    \sep \Delta , l\colonε 
  \end{eqnarray*}
\subsubsection*{Typing (extension of the previous rules)}
  \begin{mathpar}
    {\inferrule* [left=Var] {\Delta (x) = \forall \alist{χ}.\theta } {\Delta |-x~\alist{ε } :
        \theta [\alist{χ}|->\alist{κ }]}} 
    \and
    {\inferrule* [left=Rvar] {r\in\Delta (l)} {\Delta |-!\,r^l : \inttype} }
    \and
    {\inferrule* [left=EffAbs] {\Delta ,l\colonε |-f:\theta }
      {\Delta |-\lambda l \colon  ε .f : ε  -> \theta } }
    \and
    {\inferrule* [left=EffFA] {\Delta ,l\colonε |-f:\proptype}
      {\Delta |-\forall l\colon ε .f : \proptype} }
    \and
    {\inferrule* [left=EffRec] 
      {\Delta|-f:ε ->\theta  \\ \Delta |-f_i :
        \inttype \\ \ev_j\in \Delta (l_j) \\ ε =
        \alist{r_i}\alist{\ev_j} } 
      {\Delta |-f~\{r_i = f_i,\ev_j = l_j\}: \theta } }
  \end{mathpar}

In the following, we denote $σ$ an atomic substitution either of the
form $[\ev|->ε] $ or of the form $[ρ|->r]$.
\begin{definition}
  Let $σ$ be a substitution. Then the application of this substitution
  is defined differently for effect substitutions and reference
  substitutions. If $σ = [\ev|->ε_0]$, then
  \begin{equation*}
    s\sigma= \left(\bigoplus_{b\in
                  ε_s,b\neq\ev}s(b)\sigma~\oplus
                ε_0^{s(\ev)}\right).
  \end{equation*}
  In the other case of $σ=[ρ|->r]$, 
  \begin{equation*}
    %TODO introduce notation for that
    s\sigma= \bigoplus_{b\in ε} bσ =  s(b)σ.
  \end{equation*}
    
\end{definition}
\begin{definition}
  The application of a substitution $\sigma$ to a formula is defined in the
  following way (we only show the cases for the additional constructs,
  otherwise the definition is straightforward, descending recursively
  into subterms):
  \begin{eqnarray*}
    (\lambda l\colonε.f)\sigma &=& \lambda l\colon(ε\sigma).(f\sigma)\\
    (\forall l\colonε.f)\sigma &=& \forall l\colon(ε\sigma).(f\sigma)\\
    (f~s)σ &=& fσ~sσ\\
    (!\,r)σ &=& !\,(rσ)
  \end{eqnarray*}
\end{definition}

\begin{definition}
  %TODO polish
  We define the application of a label substitution $[l|->s_l]$.
  \begin{description}
  \item[for records:] $s[l|->s_l] = \bigoplus_{b\inε}s(b)$
  \end{description}
\end{definition}

\begin{prop}
  For two effect records $s_1,s_2$ of domain $ε_1$ and $ε_2$,
  respectively, and atomic substitution $σ$, we have
  \begin{equation*}
    (s_1\oplus s_2)σ \equiv s_1σ \oplus s_2σ.
  \end{equation*}
  \label{prop:recsubst}
\end{prop}
\begin{proof}
  The proof is trivial for the reference substitution case. In the
  case of an effect substitution, let $σ=[\ev|->ε_0]$. Then we write
  the following chain of equations.
  \begin{align*}
  s_1 σ\oplus s_2σ &= 
  \bigoplus_{b\inε_1σε_2σ}\left\{
    \begin{array}[]{ll}
      s_2σ(b) & \text{if}~ b\inε_2σ\\
      s_1σ(b) & \text{otherwise}
    \end{array}
\right.\\
  &=\bigoplus_{b\inε_1σε_2σ}\left\{
    \begin{array}[]{ll}
      (\bigoplus_{b\inε_1} s_2(b)σ \oplus ε^{s_2(\ev)})(b) & \text{if}~ b\inε_2\\
      (\bigoplus_{b\inε_2} s_1(b)σ \oplus ε^{s_1(\ev)})(b) &
      \text{otherwise}\
    \end{array}
\right.\\
  &=\left(\bigoplus_{b\inε_1σε_2σ}\left\{
    \begin{array}[]{ll}
      (\bigoplus_{b\inε_1} s_2(b)σ )(b) & \text{if}~ b\inε_2\\
      (\bigoplus_{b\inε_2} s_1(b)σ )(b) &
      \text{otherwise}\
    \end{array}
\right.\right)
\oplus ε^{(s_1\oplus s_2)(\ev)}\\
  &=\left(\bigoplus_{b\inε_1σε_2σ}\left\{
    \begin{array}[]{ll}
      s_2(b)σ & \text{if}~ b\inε_2\\
      s_1(b)σ &
      \text{otherwise}\
    \end{array}
\right.\right)
\oplus ε^{(s_1\oplus s_2)(\ev)}\\
  &=\left(\bigoplus_{b\inε_1σε_2σ}\left\{
    \begin{array}[]{ll}
      s_2(b) & \text{if}~ b\inε_2\\
      s_1(b) &
      \text{otherwise}\
    \end{array}
\right.\right)σ\\
&=(s_1\oplus s_2)σ
  \end{align*}
  This proof silently supposes that both effect records contain the
  field $\ev$. If one defines $ε^{s(\ev)}$ to be the empty record if
  $s$ does not contain $\ev$, then the same proof is true in the other
  three cases.
\end{proof}

\subsubsection*{Interpretation of the new constructs}

\begin{eqnarray*}
  [|\lambda l\colon ε.f|] & = & \lambda _{b\inε }(b^l\colon \theta _b).[|f|]\\
  ~[|\forall l\colon ε.f|] & = & \forall _{b\in ε
  }(b^l\colon \theta _b). [|f|]\\
  ~[|!\,r^l |] & = &r^l \\
  ~[|f~s~|] & = &
          [|f|]~[|f_1|]~\cdots~[|f_n|]~\ev_1^{l_1}~\cdots~\ev_m^{l_m}
\end{eqnarray*}
where
\begin{equation*}
s = \{r_1 = f_1,\cdots,r_n = f_n,\ev_1 = l_1,\cdots ,\ev_m
  = l_m \}
\end{equation*}

\subsubsection*{Equational Theory}

In addition to the equation schemes present in SHOL, we add the
following equation schema:
\begin{equation}
  (λl:ε.f)~s = fμ,\qquad μ = [ b^l|->s(b) \mid b\inε].
  \label{eq:eqschema}
\end{equation}

\subsubsection*{Effect substitutions}

\begin{definition}
  An effect substitution $σ = [\ev |-> ε]$ is {\em compatible} with an
  effect $ε'$ if $\ev\inε$ implies $ε$ disjoint from
  $ε'$. Furthermore, a reference substitution $[ρ|->r]$ is {\em
    compatible} with $ε'$ if $ρ\inε'$ implies $r\notinε'$.
\end{definition}
\begin{definition}
  A substitution $σ$ is {\em compatible } with a formula $f$
  if $σ$ is compatible with every effect expression and effect record
  in $f$.
\end{definition}
\begin{definition}
  Let $μ = [Λ\alist{χ}.x |-> f_x]$. Then $μ$ is {\em compatible} with $f$
  iff for any $x~\alist{κ}$ in $f$, $[\alist{χ} |-> κ ]$ is
  compatible with $f_x$.
\end{definition}

\subsubsection*{Soundness of the translation to SHOL}

\begin{prop}
  We have $s(r)σ = sσ(rσ)$ for any compatible substitution
  $σ$ and any effect record $s$ whose domain contains $r$.
  \label{prop:effrecsubstswap}
\end{prop}
\begin{proof}
  Let $σ=[ρ|->r_0]$. Then
    \begin{equation*}
      s(r)σ = \left(\bigoplus_{b\inε}bσ = s(b)σ\right)(rσ) = sσ(rσ)
    \end{equation*}
  On the other hand, if $σ=[\ev|->ε_0]$, then
  \begin{equation*}
  s(r)σ = (\bigoplus_{b\inε}s(b)σ \oplus ε_0^{s(\ev)})(r) = sσ(rσ)
  \end{equation*}
  We have used the fact that $ε_0$ does not contain $r$ (because it is
  compatible with the domain of $s$, and that $rσ=r$ for effect
  substitutions.
\end{proof}

\begin{definition}
  Let $f$ be an effect formula and $μ$ be a compatible
  value substitution. Then write $ μ||=f$ for $||= fμ$, or, equivalently, $
  |=[|fμ|]$. Thus, $μ$ contains the only free variables of $f$.
\end{definition}

\begin{lem}
  Let $s_0$ be an effect record of domain $ε_0$ such that 
  $Δ|-_r s_0 : ε_0$. Let $Δ,l:ε_0|-f:θ$ and $Δ,l:ε_0|-_rs : ε$. Also,
  let 
  \begin{eqnarray*}
μ &=& [b^l|->s_0(b)\mid b\in ε_0] \\
μ'&=&[b^l|->s_0σ(b)\mid b\inε_0σ]
  \end{eqnarray*}
  where $σ$ is some substitution compatible with $ε,ε_0$ and $f$. We have
  \begin{enumerate}
    \item $fμσ == fσμ'$,
    \item $sμσ == sμ'σ$.
  \end{enumerate}
  \label{lem:substequal}
\end{lem}
\begin{proof}
  We have to prove both statements using mutual induction over the
  structure of formulas and effect records. Let us start with the
  first one We start by proving the first statement. The only
  interesting cases are the cases where the substitutions come into
  play. This concerns effect abstraction, quantification, effect
  records and references to variables.
  \begin{description}
  \item[Case $!\,r^l$.]
    We can write the following equation:
    \begin{align*}
      (!\,r^l)μσ &== s(r)σ ==sσ(rσ) == (!\,(rσ)^l)μ' == (!\,r^l)σμ'
    \end{align*}
    We have used proposition~\ref{prop:effrecsubstswap}.
  \item[Case $λl:ε_1.f_1$.] We write the following chain of equations.
    \begin{align*}
      (λl_1:ε_1.f_1)μσ&== (λl_1:ε_1.f_1μ)σ\\
      &== λl_1:ε_1σ.f_1μσ\\
      &==λl_1:ε_1σ.f_1σμ'\\
      &==(λl_1:ε_1σ.f_1σ)μ'\\
      &==(λl_1:ε_1.f_1)σμ'
    \end{align*}
    We have used the induction hypothesis, and the fact that the
    variable substitutions $μ,μ'$ simply descend to the body of the
    abstraction.
  \item[Case $∀l:ε_1.f_1$]. Analogous.
  \item[Case $f_1~s_1$.]
    We have $(f~s)μσ = fμσ~sμσ = fσμ'~sσμ' =(f~s)σμ' $ by the induction
    hypotheses.
  \end{description}
  Now, we need to prove the second part of the claim. We distinguish
  the two different kinds of substitutions: for effects and for references.
  \begin{description}
  \item[Case $σ=\lbracket\ev|->ε_σ\rbracket$.] 
    We write
    \begin{align*}
       sμσ &== \left(\bigoplus_{b\in ε} s(b)μ\right)σ\\
      &==\bigoplus_{b\in ε} s(b)μσ \oplusε_σ^{s(\ev)μ} \\
      &==\bigoplus_{b\in ε}s(b)σμ'\oplus \bigoplus_{b\inε_σ}b^{s(\ev)}μ'\\
      &==\left(\bigoplus_{b\inε_1σ}s(b)σ\oplusε_0^{s(\ev)}\right)μ' \\
      &== sσμ'
    \end{align*}
    This step from the second to the third line requires some
    explanations. First, we have used the induction hypothesis for the
    references part of s. Second, let us give the proof that the right
    hand sides of $\oplus$ are indeed equal. For this, first suppose
    that $s(\ev) = l$. 
    \begin{description}
    \item[Subcase $\ev\inε_0$.] Then 
    \end{description}

Then we obtain the equality
    \begin{equation*}
      ε_σ^{s_0(\ev)} = \bigoplus_{b\inεσ}s_0(b)σ.
    \end{equation*}
    However, $σ$ does never change the term on the right, as
    $\ev\notin εσ$. Thus, the equation is true by definition. In the
    other case $s(\ev) \neq l$, we have 


    However, the
    induction hypothesis only accounts for the concrete reference
    fields of $s_1$. For the others (effect variable fields), consider
    some effect variable $\ev_1\in dom(s_1) = ε_1$. We know that
    $\ev_1\notin ε_0$, because σ is compatible with $f$.  First
    suppose $\ev_1\inε$. Then the identity becomes $s(\ev_1)== 
    s(\ev_1)$, which is trivially true.  On the other hand, if
    $\ev_1\notin ε$, the substitutions do not apply, and we obtain
    $s_1(\ev_1) == s_1(\ev_1)$.
  \end{description}
\end{proof}

\begin{lem}[Preservation of Equality under Substitution] In effect
  logic, $μ||= f_1 = f_2$ implies $μ||=f_1σ = f_2σ$, for any (atomic)
  substitution which is compatible with $f_1,f_2$.
\end{lem}
\begin{proof}
  By induction over the last rule applied to derive the equality.
  The only case to consider is the case when the last rule applied is
  the equation scheme~(\ref{eq:eqschema}); all other
  cases do not involve effect abstraction. This means that the
  equation is of the form
\begin{equation*}
  (λl:ε.f)~s = fμ,\qquad μ = [ b^l|->s(b) \mid b\inε].
\end{equation*}
We now want to prove
\begin{equation*}
  ((λl:ε.f)~s)σ = (fμ)σ
\end{equation*}
for some substitution $σ=[\ev|->ε_0]$. We can write
\begin{align*}
  ((λl:ε.f)~s)σ &= (λl:εσ.fσ)~sσ \\
      &= fσμ'\quad\text{where}~μ' = [b^l|->sσ(b)\mid b\in εσ]\\
      &== fμσ\quad\text{by Lemma~\ref{lem:substequal}}
\end{align*}
\end{proof}

\begin{lem}[Preservation of Equality under Translation]
  $f = f'$ in effect logic implies $[|f|] = [|f'|]$ in SHOL.
\end{lem}
\begin{proof}
  A straightforward induction over the derivation of the proof of $f =
  f'$. In the case where the schema (\ref{eq:eqschema}) has been
  applied, one can obtain the same equality in SHOL by applying
  $β$-equality $n$ times, where $n$ is the number of fields of the
  effect record involved.
\end{proof}

In summary, we have shown that equation schema~(\ref{eq:eqschema}) is
actually {\em admissible} in effect logic, if one defines validity in
effect logic by the validity of the translation to SHOL, as we do. No
additional equation can be proved. Using the lemma about preservation
of equality under substitution, we can also state the following
theorem:
\begin{thm}
  For any $σ = [\ev |-> ε]$ and $f$ such that $σ$ is compatible with
  $fμ$ and $fμ$ monomorphic, we have $μ||=∀\ev.f$ implies $μ||=fσ$.
  \label{thm:formulasubstvalid}
\end{thm}

This theorem is essential to our calculus, as it directly applies to
the proof obligations for correctness of effect polymorphic
functions. It states that the correctness remains valid if one
instantiates the function with a (compatible) effect substitution. Our
typing constraints guarantee that no other effect instantiation can
occur.

\subsection{Lifting Values and Types}

\begin{definition}
  We define a function $\lceil \cdot \rceil$ from program values to logic
  expressions as follows:
  \begin{eqnarray*}
    \ceil{c} &=& c \\
    \ceil{x~\alist{ε }} &=& x~\alist{ε } \\
    \ceil{\lambda x\colon τ .\{p\}e\{q\}}  &=& (\lambda x:\ceil{τ }.p,\lambda x:\ceil{τ }.q)\\
    \ceil{\rec~f~(x:τ).\{p\}e\{q\}} &=&
    (\lambda x:\ceil{τ }.p,\lambda x:\ceil{τ }.q)
  \end{eqnarray*}
\end{definition}
\begin{definition}
  We define a function $\lceil \cdot \rceil$ from program types to logic types
  as follows:
  \begin{eqnarray*}
    \ceil{\iota } &=& \iota  \\
    \ceil{τ ->^ε τ '} &=& (\ceil{τ } -> ε
    -> \proptype) \times\\
    & & (\ceil{τ } -> ε  -> ε  -> \ceil{τ '}
    -> \proptype)\\
  \ceil{\forall \alist{χ}.τ } & =& \forall \alist{χ}.\ceil{τ }
  \end{eqnarray*}
\end{definition}
\begin{prop}
The following two simple properties are trivially verified.
  \begin{enumerate}
  \item 
  $\ceil{τ} [\alist{χ} |-> \alist{ε}] == \ceil{τ[\alist{χ} |-> \alist{ε}]}$
  \item
  $\ceil{v} [\alist{χ} |-> \alist{ε}] == \ceil{v[\alist{χ} |-> \alist{ε}]}$
  \end{enumerate}
  \label{prop:typevaleffsub}
\end{prop}

\subsection{Weakest preconditions}

\subsection*{Definition}
\begin{eqnarray*}
  c(x~\alist{ε } ) &==& \True \\
  c(c) &==& \True \\
  c(\lambda x\colonτ .\{p\}e\{q\}) &==& \forall x\colon\ceil{τ }.\forall
  l\colon ε.p~ε^l => \wpre_{ε ^l}(e,
q~ε^l)\\
  c(v = \rec~(y\colonτ ->^ε τ ')~(x \colon τ ).\{p\}e\{q\})
  &==& \forall y\colon\ceil{τ ->^ε τ '}.\forall
  x\colon\ceil{τ }.\forall l\colonε.p~ε^l => \\
  & & y = \ceil{v}  => \wpre_{ε ^l}(e,q~ε^l)\\
  & & \\
  \wpre_{\emptyset}(v,q) &==& q~\{\}~\ceil{v} /\ c(v) \\
  \wpre_{r^l}(!\,r,q) &==& q~\{r = r^l\}~!\,r^l \\
  \wpre_{\emptyset}(r:=v,q) &==& c(v) /\ q~\{r = \ceil{v}\}~() \\
  \wpre_{ε ^l}( v_{τ '->^ε  τ }~v', q) &==& 
  c(v) /\ c(v') /\ \pre~ \ceil{v}~ \ceil{v'}~ ε ^l /\
  \\
  & & \forall m\colonε.\forall x\colon\ceil{τ }. 
  \post~\ceil{v}~\ceil{v'}~ε ^l~ε ^m~x => q~ε ^m~x \\
  \wpre_{ε ^l}( \letst~x~\alist{χ} = v~\inst~e , q) &==& 
  \forall \alist{\ev}.c(v) /\ \wpre_{ε ^l}(e,q)[\Lambda
  \alist{\ev}. x |-> \ceil{v} ] \\
  \wpre_{(ε _1ε _2)^l}( \letst~x = {e_1}_{τ
    '}~\inst~e_2, q) &==&
  \wpre_{ε _1^l}(e_1,\lambda m:ε _1.\lambda x\colon\ceil{τ '}.
  \wpre_{ε _2\subseteq (ε _2^l\oplusε _1^m)}(e_2,q))\\
  \wpre_{ε^l}(\ifst~v~\thenst~e_1~\elsest~e_2,q) &==& \ceil{v} = \true =>
  \wpre_{ε_1\subseteqε^l}(e_1,q) /\ \\
  & & \ceil{v} = false => \wpre_{ε_2\subseteqε^l}(e_2,q) \\
  & & \\
  \wpre_{ε _1\subseteqε ^l}(e,q) &==&
  \wpre_{ε ^l|ε _1}(e,\lambda m\colonε _1.q~(ε ^l\oplus ε _1^m))
\end{eqnarray*}

\subsection*{Soundness of the \wpre-calculus}

\begin{prop}
  If $v$ is well-typed of type $τ$ using $\alist{χ}$ and
  $σ=[\alist{χ}|->\alist{ε}]$ such that $σ\simτ$, then $σ$ is
  compatible with $c(v)$.
  \label{prop:compatvcv}
\end{prop}
\begin{proof}
  This is trivial for $v \equiv x~\alist{ε}$ and $v \equiv c$, because
  $c(v) = \True$ in this case.  Now if $v == λx:τ.\{p\}e\{q\}$, then
  $τ= τ_1->^ετ_2$ and because of $σ\simτ$, $σ$ is compatible with
  $ε$. The formula
  \begin{equation*}
        c(v)== ∀x:\ceil{τ}.∀l:ε.p => \wpre_{ε^l}(e,λm:ε.q)
  \end{equation*}
  can only contain effects not larger than $ε$ (because of the side
  conditions when typing the $λ$-abstraction), and thus the claim is
  true. The case for $\rec$ is analogous.
\end{proof}

\begin{prop}
  For any (not necessarily atomic nor idempotent) effect substitution
  $σ$, we have
  \begin{equation*}
    \wpre(e,q)σ \equiv \wpre(eσ, qσ)
  \end{equation*}
  and
  \begin{equation*}
    c(v)σ \equiv c(vσ) 
  \end{equation*}
  \label{prop:wpceffsubeq}
\end{prop}
\begin{proof}
  Straightforward using prop.~\ref{prop:recsubst}.
\end{proof}

\begin{rem}
  Proposition \ref{prop:wpceffsubeq} together with Theorem
  \ref{thm:formulasubstvalid} gives us the following fact: If
  $∀\alist{χ}.c(v)$ is valid and $σ$ is compatible with
  $c(v)$, then $c(vσ)$ is valid as well.
  \label{rem:correcteffsub}
\end{rem}

The following is the main lemma. It plays a similar rôle as the
substitution lemma in subject reduction proofs. Let us introduce the
following convention: if we use effect substitution or generalized
value substitution, as in $fσ$ or $fμ$, we assume (or have to prove)
that $σ$, respectively $μ$, is compatible with $f$. We assume a
similar statement for $μ||=f $, where $μ$ must be compatible with $f$
and $[|fμ|]$ be monomorphic.
 
\begin{lem}
  Let $e$ be an expression, and let $ε$ and $τ$ be its effect and
  type.  Let $μ = [Λ\alist{χ}.y |-> v']$, $q,q'$ be formulas of type
  $ε->\ceil{τ}->\proptype$, and set $θ = \ceil{τ}$. Suppose for some
  $μ'$ we have
  the following premises:
  \begin{itemize}
  \item $μ'||=∀\alist{χ}.c(v') $;
  \item $y$ does not occur in $q'$;
  \item $μ'||=∀x:θ.∀l:ε. qμ~ε^l~x => q'~ε^l~x$
  \end{itemize}
  Then we have:
  \begin{itemize}
    \item $μ'||=∀l:ε.\wpre_{ε^l}(e,q)μ => \wpre_{ε^l}(eμ,q')$
    \item $μ'||=c(v)μ => c(vμ)$ 
    \item $μ'||=∀l:ε.\wpre_{ε'\subseteqε^l}(e,q)μ => \wpre_{ε'\subseteqε^l}(eμ,q')$
  \end{itemize}
  \label{lem:effsubweakeningwp}
\end{lem}
\begin{rem}
  \label{rem:effsubweakeningwp}
  Actually, we want to prove two simpler properties (with identical
  assumptions on $μ,v',q$ and $q'$):
  \begin{enumerate}
  \item $μ'||=∀x:θ.∀l:ε.q~ε^l~x => q'~ε^l~x$ implies $μ'||=∀l:ε.\wpre_{ε^l}(e,q) =>
    \wpre_{ε^l}(e,q')$ ;
    \item $μ'||=∀l:ε.\wpre_{ε^l}(e,q)μ => \wpre_{ε^l}(eμ,qμ)$.
  \end{enumerate}
  The first states that the weakest precondition of a stronger
  statement implies the one of a weaker statement; the second one
  states that the weakest precondition calculus stays valid if one
  substitutes a (correct) value for a variable. Both properties are
  necessary for the $\wpre$ calculus to be modular.

  The formulation of the $\wpre$ calculus, however, forces us to prove
  the two properties at the same time, hence Lemma
  \ref{lem:effsubweakeningwp}. It is easy to see that the two desired
  properties are implied by this lemma: Setting $μ = id$ we obtain the
  first property, and setting $q' = qμ$, we obtain the second. 
\end{rem}

\begin{proof}
  The third part of the lemma follows easily from the first, by
  expanding definitions. We will use this property, however, in some
  cases for the induction hypotheses of the first property.
  
  Let's prove the second part about correctness of values. In the case of
  constants, there is nothing to prove.
  \begin{description}
    \item[Case $λ$] 
      \begin{align*}
        μ'||=c(λx:τ.\{p\}e\{q\})μ ~&<=>~μ'||= (∀x:\ceil{τ}.∀l:ε. p~ε^l =>
        \wpre_{ε^l}(e,q~ε^l))μ \\
        & <=>~ μ'||= ∀x:\ceil{τ}.∀l:ε.pμ~ε^l => \wpre_{ε^l}(e,q~ε^l)μ \\
        & =>~μ'||= ∀x:\ceil{τ}.∀l:ε.pμ~ε^l => \wpre_{ε^l}(eμ,qμ~ε^l) \\
        &<=>~μ'||= c(λx:τ.\{pμ\}eμ\{qμ\})
      \end{align*}
      The step from line 2 to line 3 is possible because of the induction
      hypothesis on $\wpre(e,q)$ (see remark \ref{rem:effsubweakeningwp}). 
    \item[Case $\rec$] This case is entirely analogous.
      
    \item[Case $x~\alist{ε}$] If $x$ is different from $y$, there is
      nothing to prove. Otherwise, set $σ = [\ev|->ε]$. As
      $x~\alist{ε}$ is well-typed, we know that $x$ is of type
      $∀\ev.τ'$ such that $σ\simτ$. Now $v$, under the assumption
      $\alist{χ}$, must be of type $τ$. We thus know by proposition~
      \ref{prop:compatvcv} that $c(v)$ is compatible to $σ$. We have
      $c((x~\alist{ε}))= c(vσ) = c(v)σ$ (proposition
      \ref{prop:wpceffsubeq}) and we can apply theorem
      \ref{thm:formulasubstvalid} to conclude.
  \end{description}
  
  Now let's prove the part about \wpre.
  \begin{description}
    \item[Case $v$]
      \begin{align*}
        μ'||= \wpre_{ε^l}(v,q)μ~ &<=>~μ'||= (q~\{\}~\ceil{v})μ /\ c(v)μ\\
        & <=>~μ'||= qμ~\{\}~\ceil{v}μ /\ c(v)μ\\
        & => ~μ'||=q'~\{\}~\ceil{vμ} /\ c(vμ) \\
        &<=>~ μ'||=\wpre_{ε^l}(vμ,q')
      \end{align*}
      We can go from the second to the third line by using proposition
      \ref{prop:typevaleffsub}, the induction hypothesis on values and the
      hypothesis on the implication from $q$ to $q'$.

    \item[Case $!\,r$]  
      \begin{align*}
        μ'||= \wpre_{ε^l}(!\,r,q)μ &<=>~μ'||= qμ~\{\}~r^l\\
        & =>~μ'||= q'~\{\}~r^l\\
        & <=>~μ'||= \wpre_{ε^l}(!\,r,q')
      \end{align*}
      We have only used the condition on $q$ and $q'$.

    \item[Case $r:=v$ ] 
      \begin{align*}
        μ'||= \wpre_{ε^l}(r:=v,q)μ &<=>~μ'||= c(v)μ /\ qμ~\{r = \ceil{v}μ\}~\void\\
        & =>~μ'||= c(vμ) /\ q'~\{r = \ceil{vμ}\}~\void\\
        & <=>~μ'||= \wpre_{ε^l}(r:=vμ,q')
      \end{align*}
      We used the induction hypothesis on $c(v)$ and the implication
      between $q$ and $q'$.

   \item[Case app] This case is very similar to the cases before (but
     the manipulated formula is more complex), and the induction
     hypothesis on $c(v)$ is used twice.
   \item[Case monomorphic $let$ ] 
     Here we reason inside out: 
     \begin{multline*}
       μ'||= ∀x:θ.∀l:(ε_1ε_2). qμ~(ε_1ε_2)^l~x => q'(ε_1ε_2)^l~x\\
       =>μ'||=
       ∀l:(ε_1ε_2).∀ε_1^m.∀ε_2^o.∀z:θ.(λo:ε_2.qμ~(ε_1^m\oplusε_2^o))~ε_2^l~z
       => (λo:ε_2.q'~(ε_1^m\oplusε_2^o))~ε_2^l~z\
     \end{multline*}
     Now set $f(q) = λo:ε_2.q~(ε_1^m\oplusε_2^o)$ (note that $f(qμ) =
     f(q)μ$), so we can continue:
     \begin{equation*}
       μ'||=∀l:(ε_1ε_2).∀m:ε_1.\wpre_{(ε_2^l\oplusε_1^m)|_{ε_2}}(e_2,f(q))μ
       =>\wpre_{(ε_2^l\oplusε_1^m)|_{ε_2}}(e_2μ,f(q'))
      \end{equation*}
      Now, set
      \begin{equation*}
        g(e,q) =
        λm:ε_1.λx:\ceil{τ'}.\wpre_{(ε_2^l\oplusε_1^m)|_{ε_2}}(e,f(q))
        = \wpre_{ε_2\subseteq (ε_2^l\oplusε_1^m)}(e,q)
      \end{equation*}
      and we have
     \begin{align*}
       &μ'||=∀l:(ε_1ε_2).∀z:\ceil{τ'}.g(e_2,q)μ~ε_1^l~z => g(e_2μ,q')~ε_1^l~z\\
       =>~&μ'||=∀l:(ε_1ε_2).\wpre_{ε_1^l}(e_1,g(e_2,q))μ => \wpre_{ε_1^l}(e_1μ,g(e_2μ,q'))\\
       =>~&μ'||=∀l:(ε_1ε_2).\wpre_{ε_1^l}(\letst~x=e_1~\inst~e_2,q)μ => 
         \wpre_{ε_1^l}(\letst~x=e_1μ~\inst~e_2μ,q')
     \end{align*}
     We have used the induction hypothesis twice here, for $e_1$ and
     $e_2$ and for different formulas.
     
   \item[Case polymorphic $let$ ] 
     Let us write $μ_0$ for $[Λ\alist{χ}.x |-> v] $, $μ_0'$ for
     $[Λ\alist{χ}.x |-> vμ] $ and $μ_0μ$ for the combination of the two
     substitutions. Now,
     \begin{align*}
       μ'||= \wpre_{ε^l}(\letst~x~\alist{χ} = v : τ'~\inst~e,q)μ
       &<=>~μ'||= ∀\alist{χ}.c(v)μ /\ \wpre_{ε^l}(e,q)μ_0μ\\
         &<=>~μ'||= ∀\alist{χ}.c(v)μ /\ \wpre_{ε^l}(e,q)μμ_0'\\
         &=>~μ'||= ∀\alist{χ}.c(vμ) /\ \wpre_{ε^l}(eμ,q')μ_0'\\
         &<=>~μ'||= \wpre_{ε^l}(\letst~x~\alist{χ} = vμ : τ'~\inst~eμ,q')
     \end{align*}
     In the third line we can exchange the substitutions in this way because
     $v'$ does not contain $x$. In the next line, we apply the
     induction hypothesis, but on $μ'μ_0'||= \wpre_{ε_0}(e,q)μ$.
     \item[Case $if$] We have
       \begin{align*}
         &μ'||=\wpre_{ε^l}(\ifst~v~\thenst~e_1~\elsest~e_2,q)μ\\
         <=>& μ'||= \ceil{v}μ = true =>
         \wpre_{ε_2\subseteqε^l}(e_2,q)μ/\ 
         \ceil{v}μ = false => 
         \wpre_{ε_2\subseteqε^l}(e_2,q)μ\\
         =>& μ'||= \ceil{vμ} = true
         =>\wpre_{ε_2\subseteqε^l}(e_2μ,q')/\
         \ceil{vμ}
         = false =>
         \wpre_{ε_2\subseteqε^l}(e_2μ,q')\\
         <=>& μ'||= \wpre_{ε^l}((\ifst~v~\thenst~e_1~\elsest~e_2)μ,q')
       \end{align*}
  \end{description}
\end{proof}

Let us now proceed to the subject reduction theorem. This kind of
theorem always states that some property (typically the type of an
expression in type safety proofs) remains valid when reducing a term
using the reduction relation. Here, it's the validity of the
preconditions that is preserved; if the weakest precondition $p$ of $e$
and $q$ is valid at some point in the reduction process, the weakest
precondition $p'$ of $e'$ and $q$ after reduction is valid as well.

Let us define $ε_s$ (where $ε$ contains only references) as the effect
record of type $ε$ in which every reference of $ε$ is set to its
value according to $s$. Now define $μ,s||=_{ε} q$, where $q$ is of type
$ε->\proptype$, as $μ||=q~ε_s$. In the particular case of
$q=\wpre(e,q)$, this gives us: 
\begin{equation*}
μ,s||=_{ε}\wpre(e,q)~ <=>~μ||=(λl:ε.\wpre_{ε^l}(e,q))~ε_s~<=>~μ||=\wpre_{ε_s}(e,q). 
\end{equation*}
Therefore, in the proofs concerning the $\wpre$ calculus, we will use
the names of the references to represent their value in s.

In the following, we also use the notation $\wpre_{ε_1\subseteq
  ε_2}(e,q)$ (note the subtle difference with $\wpre_{ε_1\subseteq
  ε_2^l}(e,q)$, the difference is that the former contains two
effects, while the latter contains one effect and one effect
record). It is defined as
\begin{equation*}
  \wpre_{ε_1\subseteq ε_2}(e,q) = λl:ε_2.\wpre_{ε_1\subseteq
    ε_2^l}(e,q).
\end{equation*}

\begin{thm}[Subject Reduction]
  For any well-typed $e$ of effect $ε$, such that $s,e --\ s',e'$ and
  $μ,s||={_ε} \wpre(e,q)$, we have $μ,s'||=_{ε} \wpre_{ε'\subseteqε}(e',q)$,
  where $ε'$ is the effect of $e'$.
  \label{thm:basered}
\end{thm}
\begin{proof}
  By induction over the reduction of $s,e$.
  \begin{description}

  \item[Case $!\,r$] Let $s(r) = n$. In this case, $μ,s||=_r\wpre(!\,r,q)
    <=> μ||=q~\{r = n\}~n <=> μ,s||=_r\wpre_{∅\subseteq r}(n,q)$. 

  \item[Case $r:=n$ ] $s,r:= n --\ s[r|-> n], \void $. We have
    $μ,s||=_r\wpre(r:=n,q) <=> μ||= c(v) /\ q~\{r=n\}~\void$ and
    $\wpre_{∅\subseteq r}(\void,q) = q~\{r=n\}~\void $ is identical.

    \item[Case $let$ without generalization] 
      Suppose first that no effect generalization takes place.
      Then $s, \letst~x = v~\inst~e --\ s, e[x|->v]$. We also have 
      \begin{align*}
      \wpre_{ε^l}(\letst~x = v : τ~\inst~ e , q) &=
  \wpre_{∅}(v,λm:().λx:\ceil{τ'}.\wpre_{ε^l}(e_2, q))\\
            &= c(v) /\  \wpre_{ε^l}(e_2,q)[x|->\ceil{v}]
      \end{align*}
      Using Lemma \ref{lem:effsubweakeningwp} and the fact that
      $q[x|->[v] ] => q $ (because $q$ does not contain $x$), this can
      also be written as $\wpre(e[x|->v],q)$. 
      
    \item[Case $let$ with generalization] 
      We have 
      \begin{equation*}
      s, \letst~x~\alist{χ} = v~ \inst~ e --\ s, e[Λ\alist{χ}.x|->\ceil{v}] . 
      \end{equation*}
      We also have 
      \begin{equation*}
        \wpre_{ε^l}(\letst~x~\alist{χ} = v~\inst~e,q) = ∀\alist{χ}.c(v) /\ 
        \wpre_{ε^l}(e,q)[Λ\alist{χ}.x|->v ].
      \end{equation*}
      For the same reason as above, we have $\wpre(e[Λ\alist{χ}.x|->v],q)$,
      which is the desired formula. The substitution is compatible by
      definition of the $\wpre$ calculus.

    \item[Case $β$] In the following, let $v' = λy:τ.\{p'\}e\{q'\}$ and
      suppose that the result of application is of type $τ'$.
      We have $s, v'~v --\ s, e[y|-> v] $, and $s||= \wpre(v'~v, q)$.
      We want to obtain $s||=\wpre(e[y|->v],q)$.
       First, we have 
       \begin{equation}
         ||= p'[y|-> \ceil{v} ]~ε_s,
         \label{eq:beta}
       \end{equation}
       as the precondition of $v'$ applied to the argument. Now, the
       correctness of $v'$ is written
       \begin{equation*}
         c(v') = ∀l:ε.∀y:\ceil{τ}. p'~ε^l => \wpre_{ε^l}(e,q'~ε^l)
       \end{equation*}
       instantiating with $ε_s$ and $\ceil{v}$ and using
       \ref{eq:beta}, we derive 
       \begin{equation}
         ||= \wpre_{ε_s}(e,q'~ε_s)[y|->\ceil{v}]
         \label{eq:subst}
       \end{equation}
       We also have
       \begin{equation*}
         ∀m:ε.∀x:\ceil{τ'}.q'[y|->~\ceil{v}]~ε_s~ε^m~x => q~ε^m~x
       \end{equation*}
       By Lemma \ref{lem:effsubweakeningwp} and \ref{eq:subst}, we can conclude
       \begin{equation*}
         \wpre_{ε_s}(e[y|->\ceil{v}],q).
       \end{equation*}
       
     \item[Case $\rec$] Let $v' = \rec~(z:τ'->τ)~(y:τ').\{p\}e\{q\}$.
       The result of the application is of type $τ$. As in the $β$
       case, we have 
       \begin{equation}
         ||= p'[y|-> \ceil{v} ]~ε_s,
         \label{eq:recbeta}
       \end{equation}
       and we also have the correctness of both values, in particular
       for the recursive function:
       \begin{equation*}
         c(v') = ∀l:ε.∀z:\ceil{τ'->τ}.∀y:\ceil{τ'}. p'~ε^l => z = \ceil{v'} => \wpre_{ε^l}(e,q'~ε^l)
       \end{equation*}
       Instantiating with $ε_s$ for $ε_0$, $v'$ for $z$ and $v$ for
       $y$, we obtain again
       \begin{equation*}
         ||= \wpre_{ε_s}(e,q'~ε_s)[y|->\ceil{v}]
       \end{equation*}
       and we can proceed as in the case for $β$.
     \item[Case $\mathit{if}$] One of the assertions $\ceil{v} = \true$ and
       $\ceil{v} = \false$ is trivially true in both cases. In either
       case, we obtain precisely the desired statement by applying
       modus ponens.
  \end{description}
\end{proof}

\begin{thm}
  For any well-typed $e$ of effect $ε$, such that $s,e --> s',e'$ and
  $s||={_ε} \wpre(e,q)$, we have $s'||=_{ε} \wpre_{ε'\subseteqε}(e',q)$,
  where $ε'$ is the effect of $e'$.
  \label{thm:onestep}
\end{thm}
\begin{proof}
  There are only two cases here: the empty context and the
  $\letst$-context. The statement about the empty context is precisely
  Theorem~\ref{thm:basered}.

  In the remaining case, we have $s,E[e]-->s',E[e']$ and thus also
  \begin{equation*}
    s,\letst~x=E[e]~\inst~e_1 --> s', \letst~x=E[e']~\inst~e_1.
  \end{equation*}
  Our induction hypothesis is that $s||=_ε\wpre(E[e],q)$ implies
  $s'||=_ε\wpre_{ε'\subseteqε}(E[e'],q)$.  We also have
  $s||=_ε\wpre(\letst~x=E[e]~\inst~e_1,q)$, which amounts to
  \begin{equation*}
    s||=_ε \wpre(E[e],λm:ε.λx:\ceil{τ'}.\wpre_{ε_2\subseteq ε_s}(e_2,q))
  \end{equation*}
  By the induction hypothesis, we have 
  \begin{equation*}
    s'||=_ε \wpre_{ε'\subseteq
      ε}(E[e'],λm:ε.λx:\ceil{τ'}.\wpre_{ε_2\subseteq ε_s}(e_2,q))
  \end{equation*}.
\end{proof}

\begin{thm}
  For any well-typed $e$ of effect $ε$, such that $s,e->>s',e'$, we
  have $s||=_ε\wpre(e,q)$ implies $s'||=_ε\wpre_{ε'\subseteq ε}(e',q)$
\end{thm}
\begin{proof}
  There are two cases. In the case of reflexivity, this is
  obvious. For transitivity, assume $s,e->>s',e'$ and
  $s',e'->>s'',e''$. Also assume that the effects of $e'$ and $e''$
  are $ε_1$ and $ε_2$, respectively. It is easy to obtain
  $s'||=_ε\wpre_{ε'\subseteq ε}(e',q)$ by one of the induction
  hypotheses. By unrolling definitions, this implies
  \begin{equation*}
s'||=_{ε_1} \wpre(e',λm:ε_1.q~(ε_{s'}\oplusε_1^m))
  \end{equation*}
  and again, by applying the induction hypothesis, we obtain
  \begin{equation*}
    s''||=_{ε_1} \wpre_{ε_2\subseteq ε_1}(e'',λm:ε_1.q~(ε_{s'}\oplusε_1^m)).
  \end{equation*}
  This transforms into the desired formula by the following chain of
  equations:
  \begin{align*}
    &s''||=_{ε_1} \wpre_{ε_2\subseteq
      ε_1}(e'',λm:ε_1.q~(ε_{s'}\oplusε_1^m))\\
    <=>&
    ||=\wpre_{ε_2\subseteq {ε_1}_{s''}}(e'',λm:ε_1.q~(ε_{s'}\oplusε_1^m)) \\
    <=>&
    ||=\wpre_{{ε_1}_{s''}|ε_2}(e'',λo:ε_2.(λm:ε_1.q~(ε_{s'}\oplusε_1^m))
    ({ε_1}_{s''} \oplus ε_2^o)) \\
    <=>&
    ||=\wpre_{{ε_1}_{s''}|ε_2}(e'',λo:ε_2.q~(ε_{s'}\oplus({ε_1}_{s''} \oplus ε_2^o))) \\
    <=> &||=\wpre_{ε_{s''}|ε_2}(e'',λo:ε_2.q~(ε_{s''}\oplus ε_2^o)) \\
    <=> &||=\wpre_{ε_2\subseteq ε_{s''}}(e'',q) \\
    <=> &s''||=_{ε}\wpre_{ε_2\subseteq ε}(e'',q) 
  \end{align*}
  From line 3 to line 4, we first exploit the fact that $ε_{s''}$
  restricted to $ε_2$ is the same as ${ε_1}_{s''}$, because of
  $ε_2\subseteq ε_1\subseteq ε$. For the same reason, and the
  additional fact that the only difference between $ε_{s'}$ and
  ${ε_1}_{s''}$ lies in the domain of $ε_2$, we can simplify the
  $\oplus$ chain to the right form.
\end{proof}

\begin{cor}
  For any $e$ and $q$ and $s,e->>s',v$, we have $s||=_ε\wpre(e,q)$
  implies $||=q~\ceil{v}~ε_{s'}$.
\end{cor}
\begin{proof}
  Direct consequence from the previous theorem and the definition of
  the $\wpre$ for values.
\end{proof}


\end{document}
