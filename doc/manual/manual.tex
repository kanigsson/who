\documentclass[a4paper]{scrreprt}

\usepackage[usenames,dvipsnames]{xcolor}
\usepackage{xcolor}
\usepackage{xspace}
\definecolor{dablue}{rgb}{0.1,0.1,0.4}

\usepackage[colorlinks=true,pdfborder={0,0,0,0},urlcolor=dablue,
linkcolor=dablue, pdftex]{hyperref}

%\newcommand{\cat}[1]{$<$\textcolor{Aquamarine}{#1}$>$}
\newcommand{\cat}[1]{$<$\hyperref[#1]{#1}$>$}
\newcommand{\likecat}[1]{$<$\textcolor{Aquamarine}{#1}$>$}
\newcommand{\catdef}[1]{\label{#1}\likecat{#1}}
\newcommand{\who}{Who\xspace}

\usepackage{alltt}

\begin{document}

\title{Who - The Manual}
\author{Johannes Kanig}

\maketitle
\tableofcontents

\chapter{The command line tool}

\chapter{A walkthrough using Arrays}

In this chapter, we aim to describe almost all syntactic elements of the
\who~language.

\section{Types, Axioms and Pure Functions}

A type definition of a polymorphic type looks as follows:
\begin{alltt}
  type array ['a||]
\end{alltt}
Notice the two bars at the end to indicate empty region and effect parameters.
Also note that despite the name, {\tt array} is {\em not} a mutable type.

We can declare functions over this type in the following way:
\begin{alltt}
  logic create ['a||] :  int -> 'a -> 'a array
  logic get ['a||] : int -> 'a array -> 'a
  logic set ['a||]: int -> 'a -> 'a array -> 'a array
  logic len ['a||] :  'a array -> int
\end{alltt}
Again, notice that these are pure functions which act on {\em functional}
arrays, for example {\tt set} returns a new, modified array.

Now, we can declare properties of these functions in form of axioms:
\begin{alltt}
  axiom update_length ['a||] :
    forall (t : 'a array) (i : int) (z : 'a).
    0 <= i \verb|/\| i < len t -> len t = len (set i z t)
  axiom get_set_neq ['a||] :
    forall (t : 'a array ) (i : int) (j : int) (z : 'a).
      (0 <= i \verb|/\| i < len t) ->
      (0 <= j \verb|/\| j < len t) ->
        i <> j -> get i (set j z t) = get i t
  axiom get_set_eq ['a||] :
    forall (t : 'a array) (i : int) (z : 'a).
      0 <= i \verb|/\| i < len t -> get i (set i z t) = z
\end{alltt}

Now, we can also {\em define} functions over those functional arrays:
\begin{alltt}
  let zeroset ['a||] (v : 'a) (a : 'a array) = set 0 v a
\end{alltt}
Here, we have defined a {\em pure} function {\tt zeroset} of type {\tt ['a||].
int -> 'a array -> 'a array}. No effects take place, and nothing has to be
proved.

\section{Functions with Annotations}

If we want to define a {\em calculatory function} which does the same, here we
go:
\begin{alltt}
  let zerosetc ['a||] (v : 'a) (a : 'a array) =
    \{ \}
    set 0 v a
    \{ \}
\end{alltt}
Notice the additional curly brackets at the beginning and the end of the
function body. The function {\tt zerosetc} has the type {\tt ['a||]. int -> 'a
array ->{} 'a array}. We see that the last (and only the last) arrow has an
effect annotation. In this case, this effect annotation is empty; no effect
takes place, because {\tt set} returns a new function.

Now, if we want to state anything about the return value of {\tt zerosetc}, we
need to write a {\em postcondition}. For example, we would like to state that
the returned array contains zero at position $i$:
\begin{alltt}
  let zerosetc ['a||] (v : 'a) (a : 'a array) =
    \{ \}
    set 0 v a
    \{ r : get 0 r = v \}
\end{alltt}
We use the syntax {\tt r : } to speak about the return value; {\tt r} is a
name chosen by the user. This definition produces a proof obligation. In this
case, this proof obligation is actually unprovable, because the array {\tt a}
might be empty, in which case the axioms do not say anything of {\tt set 0 v a}.
We have to give a {\em precondition}:
\begin{alltt}
  let zerosetc ['a||] (v : 'a) (a : 'a array) =
    \{ 0 < len a \}
    set 0 v a
    \{ r : get 0 r = v \}
\end{alltt}
And now we can prove this function.

\section{References}

To write an effectful function, we need references. A reference type on an
array is written as follows:
\begin{alltt}
  ref(r,'a array)
\end{alltt}
where {\tt r} is a region name. Regions, as types, can be concrete or
parameters of a function. Let us start with the case where it is a parameter.
We want to define the same function as before, but in an imperative way. Here
we go:
\begin{alltt}
  let zerosetc ['a|r|] (v : 'a) (a : ref(r,'a array)) =
    \{ 0 < len !!a \}
    a := set 0 v !a
    \{ get 0 !!a = v \}
\end{alltt}
There are several remarks here. First, we have to introduce the region name
used with the reference {\tt a}; this is occurrence of {\tt r} between {\tt |}
in the brackets. In programs, references are treated with {\tt :=} and {\tt !}
as usual (there is currently no special syntax for arrays). In annotations, to
access the contents of a reference, we write {\tt !!} (this is subject to
change). The return type of this function is {\tt unit} and the overall type
is:
\begin{alltt}
  ['a|r|]. 'a -> ref(r,'a array) ->{r} unit
\end{alltt}
We see that this function has an effect on {\tt r}.

Let us write a function which increments the contents of the first cell of an
array.
\begin{alltt}
  let zeroincr [|r|] (a : ref(r,int array)) =
    \{ 0 < len !!a \}
    a := set 0 (get 0 a +1) a
    \{ get 0 !!a = get 0 !!a|old + 1 \}
\end{alltt}
Here, we have used the syntax {\tt !!a|old} to speak of the {\em old} value of
{\tt a}, before the execution of the function {\tt zeroincr}.

To create references, we need an additional mechanism.
\begin{alltt}
  let zero_array [|r|] (l : int) = allocates r
    \{ 0 <= l \}
    if l == 0 then ref\{\{r\}\} ar_empty
    else ref\{\{r\}\} (create l 0)
    \{ x : forall i. 0 <= i \verb|/\| i < l -> get i !!x = 0 \}
\end{alltt}
To be able to create a reference inside this function, we need an {\tt
allocates} declaration. This enables us to use a {\em allocation capacity}
when calling {\tt ref}, using the syntax {\tt ref\{\{r\}\}}. Also notice the
use {\tt ==} instead of {\tt =} in the test, to obtain a boolean return value.

\section{Effect polymorphism}
The last missing bit is effect polymorphism. Suppose we want to implement {\tt
Array.iter} in \who. This can be easily done, at first without logical
annotations:
\begin{alltt}
let iter ['a|t|'e] (ar : ref(t, 'a array)) (f : 'a ->\{t 'e\} unit)  =
  \{  \}
  for i = 0 to len !ar - 1 do
    \{ \}
    f (get i !ar)
  done
  \{ \}
\end{alltt}
We use the syntax for {\tt for} loops here; one has to give the loop
invariant, but for the moment we leave that empty. The only other new aspect
is the use of {\em effect polymorphism}, for which is reserved the third field
of the generalisation declaration {\tt ['a|out|'e]}. We generalize over a
certain effect {\tt 'e} and say that the function argument {\tt f} has
precisely this effect {\tt 'e} (see the type of {\tt f}).

The specification of {\tt iter} is relatively complex; the reason is that any
iterative higher-order function is basically a fold, because of the unknown
effect of the function in argument.

The entire code of {\tt iter} looks like this:
\begin{alltt}
let iter ['a|t|'e]
  (inv : <t 'e> -> int ->  prop)
  (ar : ref(t, 'a array)) (f : 'a ->\{t 'e\} unit)  =
  \{ inv cur 0 \verb|/\|
    forall (i:int). 0 <= i \verb|/\| i < len (!!ar) ->
    [[ inv cur i ]] f (get i !ar) [[inv cur (i+1)]]
  \}
  for i = 0 to len !ar - 1 do
    \{ inv cur i \}
    f (get i !ar)
  done
  \{inv cur (len !!ar|old) \}
\end{alltt}
There are three new elements in this code, as well as a trick to specify this
function. First keep in mind that {\tt iter} is basically a {\tt for} loop,
this means that we need an invariant to specify it. Now, the trick is to take
this invariant as an additional argument of {\tt iter}, called {\tt inv}. The
first new element is hidden in the type of {\tt inv}, namely the type {\tt
<'e>} of its first argument. Types in angle brackets {\tt < >} represent parts
of the {\em state}, here we say that the invariant may talk about the state
represented by {\tt 'e}. Just as {\tt old} is a keyword for the iter state of
the function call, available in the postcondition, {\tt cur} is a keyword for
the current state, and it is available in any annotation.  Now the
postcondition simply states that we obtain the invariant of the final state
over the entire array. The invariant of the {\tt for} loop states that we keep
the invariant during the execution. Finally, the precondition states that the
invariant holds for the index zero, and that the invariant is preserved by the
function argument. This is expressed by the syntax
\begin{alltt}
    [[ inv cur i ]] f (get i !ar) [[inv cur (i+1)]]
\end{alltt}
which is basically a Hoare triple.

There are a few other syntactic points. First, Hoare Triples using the double
bracket syntax are syntactic sugar for a more complex predicate. Assume that {\tt
f} has the type {\tt t1 ->{e} t2} where {\tt e} is some effect, then the Hoare
triple
\begin{alltt}
    [[ p cur x ]] f x [[r : q old cur x r]]
\end{alltt}
is a shortcut for
\begin{alltt}
  forall (m n : <e>) (r : t2).
    p cur x -> pre f x m /\
    post f x m n r -> q m n x r
\end{alltt}
Here we have used the ability to quantify over the portion of a state
described by an effect expression.

Concerning state expressions, one can restrict a state expression {\tt s} to a
smaller domain described by an effect using the syntax:
\begin{alltt}
s|\{e\}
\end{alltt}

\appendix

\chapter{Concrete Syntax}

In this appendix, we describe formally the syntax of the input language.

We describe the input language by a grammar, separated in several categories.
By necessity, we have to mix concrete and abstract syntax. We use the syntax
of the form \likecat{Category} to refer to categories. Every other word or symbol
is part of the concrete syntax, with the exception of the following symbols:
\texttt{( ) + ? * |}. These symbols have the sense they usually have when
talking about grammars and regular expressions: parentheses group expressions,
{\tt *} stands for zero or more occurrences, {\tt +} for at least one
occurrence, and {\tt ?} for an optional occurrence. Finally, {\tt |}
represents a choice.  When we want to refer to the actual symbol, we write
{\tt '('}, {\tt ')'} and so on.

We write {\tt list*(expr,symbol)} or {\tt list+(expr,symbol)} to describe
lists, possibly empty or containing at least one element, of expressions
separated by a symbol. Finally, we abbreviate {\tt list+(expr, ',')} by {\tt
commasep+(expr)}.

\paragraph{}
An \catdef{Id} is program identifier or a global type and is defined as: a word
composed of {\tt [A-Za-z0-9']*}, which starts with a letter out ot {\tt
[A-Za-z]}.

\paragraph{}
A \catdef{Region} is the same category.

\paragraph{}
An \catdef{EffectVar} or \catdef{Tyvar} is defined as the same category,
but starting with a {\tt '}.

A file is a list of \cat{Declaration}s.

\paragraph{}
A \catdef{Declaration} is one of the following:
\begin{itemize}
  \item {\tt axiom} \cat{Id} \cat{Gen} {\tt :} \cat{Term}
  \item {\tt logic} \cat{Id} \cat{Gen} {\tt :} \cat{Type}
  \item {\tt type} \cat{Id} \cat{Gen}  ( = \cat{Type}){\tt ?}
  \item {\tt letregion} \cat{Id}*
  \item {\tt parameter} \cat{Id} \cat{Gen} \cat{Argslist} {\tt :} \cat{Type}
  \item {\tt parameter} \cat{Id} \cat{Gen} \cat{Argslist} {\tt :} \cat{Type}
    {\tt ,} \cat{Effect} {\tt =} \cat{Allocdef} {\cat{Pre}} {\cat{Post}}
  \item \cat{Letform}
\end{itemize}

\paragraph{}
A \catdef{Letform} is:
\begin{itemize}
  \item {\tt let} \cat{Id} \cat{Gen} \cat{Argslist} {\tt =} \cat{Term}
  \item {\tt let} \cat{Id} \cat{Gen} \cat{Argslist} {\tt =} \cat{Allocdef} {\cat{Pre}} \cat{Term} {\cat{Post}}
  \item {\tt let rec} \cat{Gen} '{\tt (}' \cat{Id} {\tt :} \cat{Type} '{\tt )}'
    \cat{Argslist} {\tt =} \cat{Allocdef} {\cat{Pre}} \cat{Term} {\cat{Post}}
\end{itemize}

\paragraph{}
A generalisation \catdef{Gen} is:
\begin{itemize}
  \item {\tt [} \cat{Tyvar}* '{\tt |}' \cat{Region}* '{\tt |}' \cat{EffectVar}*
    {\tt ]}
\end{itemize}

\paragraph{}
An argument list \catdef{Argslist} is:
\begin{itemize}
  \item ( '{\tt (}' \cat{Id} + : \cat{Type} '{\tt )}' )*
\end{itemize}

\paragraph{}
An \catdef{Allocdef} is:
\begin{itemize}
  \item ( allocates \cat{Region}* ){\tt ?}
\end{itemize}

\paragraph{}
A precondition \catdef{Pre} is:
\begin{itemize}
  \item \cat{Term}{\tt ?}
\end{itemize}

\paragraph{}
A postcondition \catdef{Post} is:
\begin{itemize}
  \item \cat{Term}{\tt ?}?
  \item (\cat{Id} {\tt :} )? \cat{Term}
\end{itemize}


\paragraph{}
A \catdef{Type} is one of the following
\begin{itemize}
  \item {\tt int}
  \item {\tt prop}
  \item \cat{Tyvar}
  \item '{\tt (}' \cat{Type} '{\tt )}'
  \item \cat{Id} \cat{Inst}?
  \item \cat{Type} \cat{Id}
  \item '{\tt (}' commasep(\cat{Type}) '{\tt )}' \cat{Id}
  \item \cat{Type} {\tt ->} \cat{Type}
  \item \cat{Type} '{\tt *}' \cat{Type}
  \item \cat{Type} {\tt ->} \cat{Effect} \cat{Type}
  \item '{\tt <}' \cat{EffectWithoutBraces} '{\tt >}'
  \item {\tt [[} \cat{Type} {\tt ]]}
  \item {\tt ref} '{\tt (}'\cat{Region} \cat{Type}'{\tt )}'
\end{itemize}

\paragraph{}
An \catdef{EffectWithoutBraces} is:
\begin{itemize}
  \item ( \cat{Id} {\tt |} \cat{EffectVar} )*
\end{itemize}

\paragraph{}
An \catdef{Effect} is:
\begin{itemize}
  \item {\tt \{} \cat{EffectWithoutBraces} {\tt \}}
\end{itemize}

\paragraph{}
An instantiation \catdef{Inst} is:
\begin{itemize}
  \item {\tt [} commasep*(\cat{Type}) '{\tt |}' \cat{Region}* '{\tt |}' \cat{Effect}* {\tt
    ]}
\end{itemize}

\paragraph{}
A \catdef{Term} is one of the following:
\begin{itemize}
  \item '{\tt (}' '{\tt )}'
  \item {\tt !} \cat{Id}
  \item {\tt !!} \cat{Id} ( '{\tt |}' \cat{Term} )
  \item \cat{Id} \cat{Inst}?
  \item '{\tt (}' \cat{Term} ( {\tt :} \cat{Type})? '{\tt )}'
  \item \cat{Term} ({\tt \{\{ } \cat{Region}* {\tt \}\}} )? \cat{Term}
  \item \cat{Term} {\tt ;} \cat{Term}
  \item \cat{Term} \cat{Infix} \cat{Term}
  \item fun \cat{Argslist} {\tt ->} \cat{Term}
  \item fun \cat{Argslist} {\tt ->} \cat{Allocdef} {\cat{Pre}} \cat{Term} {\cat{Post}}
  \item ( {\tt forall} {\tt |} {\tt exists} ) \cat{Argslist} {\tt .} \cat{Term}
  \item {\tt letregion} \cat{Region}* {\tt in} \cat{Term}
  \item {\tt if} \cat{Term} {\tt then} \cat{Term} {\tt else} \cat{Term}
  \item \cat{Letform} {\tt in} \cat{Term}
  \item {\tt for} \cat{Id} = \cat{Term} ( {\tt to | downto} ) \cat{Term} {\tt
    do} {\cat{Pre}} \cat{Term} {\tt done}
  \item {\tt [[} \cat{Pre} {\tt ]]} \cat{Term} {\tt [[} \cat{Post} {\tt ]]}
\end{itemize}

\paragraph{}
The following \catdef{Infix} symbols are predefined:
  \begin{itemize}
  \item the comparison operators of return type prop: {\tt <}, {\tt >}, {\tt
  <=}, {\tt =>}, {\tt =}, {\tt <>}
  \item the comparison operators of return type bool: {\tt <<}, {\tt >>}, {\tt
  <<=}, {\tt >>=}, {\tt ==}, {\tt !=}
  \item the logical connectives: \verb|/\|, \verb|\/|, {\tt ->}
  \item arithmetic operations: {\tt +}, {\tt -}, {\tt *}
  \item tuple construction: {\tt ,}
  \end{itemize}

\end{document}
